\documentclass[a4paper,12pt,twoside]{memoir}

% English
\usepackage[english]{babel}
\selectlanguage{english}
\usepackage[utf8]{inputenc}
\usepackage[T1]{fontenc}
\usepackage{lmodern} % Scalable font
\usepackage{microtype}
\usepackage{placeins}
\usepackage{listings}

\RequirePackage{booktabs}
\RequirePackage[table]{xcolor}
\RequirePackage{xtab}
\RequirePackage{multirow}
\PassOptionsToPackage{hyphens}{url}

% Links
\usepackage[colorlinks]{hyperref}
\hypersetup{
	allcolors = {red}
}

% Ecuaciones
\usepackage{amsmath}

% Rutas de fichero / paquete
\newcommand{\ruta}[1]{{\sffamily #1}}

% Párrafos
\nonzeroparskip

% Huérfanas y viudas
\widowpenalty100000
\clubpenalty100000

% Evitar solapes en el header
\nouppercaseheads

% Imagenes
\usepackage{graphicx}
\newcommand{\imagen}[2]{
	\begin{figure}[!h]
		\centering
		\includegraphics[width=0.9\textwidth]{#1}
		\caption{#2}\label{fig:#1}
	\end{figure}
	\FloatBarrier
}

\newcommand{\imagenflotante}[2]{
	\begin{figure}%[!h]
		\centering
		\includegraphics[width=0.9\textwidth]{#1}
		\caption{#2}\label{fig:#1}
	\end{figure}
}

% El comando \figura nos permite insertar figuras comodamente, y utilizando
% siempre el mismo formato. Los parametros son:
% 1 -> Porcentaje del ancho de página que ocupará la figura (de 0 a 1)
% 2 --> Fichero de la imagen
% 3 --> Texto a pie de imagen
% 4 --> Etiqueta (label) para referencias
% 5 --> Opciones que queramos pasarle al \includegraphics
% 6 --> Opciones de posicionamiento a pasarle a \begin{figure}
\newcommand{\figuraConPosicion}[6]{%
  \setlength{\anchoFloat}{#1\textwidth}%
  \addtolength{\anchoFloat}{-4\fboxsep}%
  \setlength{\anchoFigura}{\anchoFloat}%
  \begin{figure}[#6]
    \begin{center}%
      \Ovalbox{%
        \begin{minipage}{\anchoFloat}%
          \begin{center}%
            \includegraphics[width=\anchoFigura,#5]{#2}%
            \caption{#3}%
            \label{#4}%
          \end{center}%
        \end{minipage}
      }%
    \end{center}%
  \end{figure}%
}

%
% Comando para incluir imágenes en formato apaisado (sin marco).
\newcommand{\figuraApaisadaSinMarco}[5]{%
  \begin{figure}%
    \begin{center}%
    \includegraphics[angle=90,height=#1\textheight,#5]{#2}%
    \caption{#3}%
    \label{#4}%
    \end{center}%
  \end{figure}%
}

% Comando para incluir imágenes en formato normal (sin marco).
\newcommand{\figuraNormalSinMarco}[5]{%
  \begin{figure}%
    \begin{center}%
    \includegraphics[angle=0,height=#1\textheight,#5]{#2}%
    \caption{#3}%
    \label{#4}%
    \end{center}%
  \end{figure}%
}

% Para las tablas
\newcommand{\otoprule}{\midrule [\heavyrulewidth]}
%
% Nuevo comando para tablas pequeñas (menos de una página).
\newcommand{\tablaSmall}[5]{%
 \begin{table}
  \begin{center}
   \rowcolors {2}{gray!35}{}
   \begin{tabular}{#2}
    \toprule
    #4
    \otoprule
    #5
    \bottomrule
   \end{tabular}
   \caption{#1}
   \label{tabla:#3}
  \end{center}
 \end{table}
}

%
% Nuevo comando para tablas pequeñas (menos de una página).
\newcommand{\tablaSmallSinColores}[5]{%
 \begin{table}[H]
  \begin{center}
   \begin{tabular}{#2}
    \toprule
    #4
    \otoprule
    #5
    \bottomrule
   \end{tabular}
   \caption{#1}
   \label{tabla:#3}
  \end{center}
 \end{table}
}

\newcommand{\tablaApaisadaSmall}[5]{%
\begin{landscape}
  \begin{table}
   \begin{center}
    \rowcolors {2}{gray!35}{}
    \begin{tabular}{#2}
     \toprule
     #4
     \otoprule
     #5
     \bottomrule
    \end{tabular}
    \caption{#1}
    \label{tabla:#3}
   \end{center}
  \end{table}
\end{landscape}
}

%
% Nuevo comando para tablas grandes con cabecera y filas alternas coloreadas en gris.
\newcommand{\tabla}[6]{%
  \begin{center}
    \tablefirsthead{
      \toprule
      #5
      \otoprule
    }
    \tablehead{
      \multicolumn{#3}{l}{\small continúa desde la página anterior}\\
      \toprule
      #5
      \otoprule
    }
    \tabletail{
      \hline
      \multicolumn{#3}{r}{\small continúa en la página siguiente}\\
    }
    \tablelasttail{
      \hline
    }
    \bottomcaption{#1}
    \rowcolors {2}{gray!35}{}
    \begin{xtabular}{#2}
      #6
      \bottomrule
    \end{xtabular}
    \label{tabla:#4}
  \end{center}
}

%
% Nuevo comando para tablas grandes con cabecera.
\newcommand{\tablaSinColores}[6]{%
  \begin{center}
    \tablefirsthead{
      \toprule
      #5
      \otoprule
    }
    \tablehead{
      \multicolumn{#3}{l}{\small\sl continúa desde la página anterior}\\
      \toprule
      #5
      \otoprule
    }
    \tabletail{
      \hline
      \multicolumn{#3}{r}{\small\sl continúa en la página siguiente}\\
    }
    \tablelasttail{
      \hline
    }
    \bottomcaption{#1}
    \begin{xtabular}{#2}
      #6
      \bottomrule
    \end{xtabular}
    \label{tabla:#4}
  \end{center}
}

%
% Nuevo comando para tablas grandes sin cabecera.
\newcommand{\tablaSinCabecera}[5]{%
  \begin{center}
    \tablefirsthead{
      \toprule
    }
    \tablehead{
      \multicolumn{#3}{l}{\small\sl continúa desde la página anterior}\\
      \hline
    }
    \tabletail{
      \hline
      \multicolumn{#3}{r}{\small\sl continúa en la página siguiente}\\
    }
    \tablelasttail{
      \hline
    }
    \bottomcaption{#1}
  \begin{xtabular}{#2}
    #5
   \bottomrule
  \end{xtabular}
  \label{tabla:#4}
  \end{center}
}



\definecolor{cgoLight}{HTML}{EEEEEE}
\definecolor{cgoExtralight}{HTML}{FFFFFF}

%
% Comando para el texto en formato JSON
% Source: https://tex.stackexchange.com/questions/83085/how-to-improve-listings-display-of-json-files

% Switch used as state variable
\makeatletter
\newif\ifisvalue@json

\newcommand\jsonkey{\color{purple}}
\newcommand\jsonvalue{\color{cyan}}
\newcommand\jsonnumber{\color{orange}}
\definecolor{background}{HTML}{EEEEEE}

\lstdefinelanguage{json}{
    tabsize=4,
    showstringspaces=false,
    breaklines=true,
    keywords={false,true},
    alsoletter=0123456789.,
    morestring=[s]{"}{"},
    stringstyle=\jsonkey\ifisvalue@json\jsonvalue\fi,
    MoreSelectCharTable=\lst@DefSaveDef{`:}\colon@json{\enterMode@json},
    MoreSelectCharTable=\lst@DefSaveDef{`,}\comma@json{\exitMode@json{\comma@json}},
    MoreSelectCharTable=\lst@DefSaveDef{`\{}\bracket@json{\exitMode@json{\bracket@json}},
    basicstyle=\scriptsize\ttfamily,
    backgroundcolor=\color{background}
}

% Enter "value" mode after encountering a colon
\newcommand\enterMode@json{%
    \colon@json%
    \ifnum\lst@mode=\lst@Pmode%
        \global\isvalue@jsontrue%
    \fi
}

% Leave "value" mode: either we hit a comma, or the value is a nested object
\newcommand\exitMode@json[1]{#1\global\isvalue@jsonfalse}

\lst@AddToHook{Output}{%
    \ifisvalue@json%
        \ifnum\lst@mode=\lst@Pmode%
            \def\lst@thestyle{\jsonnumber}%
        \fi
    \fi
    % Override by keyword style if a keyword is detected
    \lsthk@DetectKeywords% 
}

\makeatother

%
% Nuevo comando para tablas grandes sin cabecera.
\newcommand{\tablaSinCabeceraConBandas}[5]{%
  \begin{center}
    \tablefirsthead{
      \toprule
    }
    \tablehead{
      \multicolumn{#3}{l}{\small\sl continúa desde la página anterior}\\
      \hline
    }
    \tabletail{
      \hline
      \multicolumn{#3}{r}{\small\sl continúa en la página siguiente}\\
    }
    \tablelasttail{
      \hline
    }
    \bottomcaption{#1}
    \rowcolors[]{1}{cgoExtralight}{cgoLight}

  \begin{xtabular}{#2}
    #5
   \bottomrule
  \end{xtabular}
  \label{tabla:#4}
  \end{center}
}


\graphicspath{ {./img/} }

% Capítulos
\chapterstyle{bianchi}
\newcommand{\capitulo}[2]{
	\setcounter{chapter}{#1}
	\setcounter{section}{0}
	\chapter*{#2}
	\addcontentsline{toc}{chapter}{#1. #2}
	\markboth{#2}{#2}
}

% Apéndices
\renewcommand{\appendixname}{Appendix}
\renewcommand*\cftappendixname{\appendixname}

\newcommand{\apendice}[1]{
	%\renewcommand{\thechapter}{A}
	\chapter{#1}
}

\renewcommand*\cftappendixname{\appendixname\ }

% Formato de portada
\makeatletter
\usepackage{xcolor}
\newcommand{\tutor}[1]{\def\@tutor{#1}}
\newcommand{\course}[1]{\def\@course{#1}}
\definecolor{cpardoBox}{HTML}{E6E6FF}
\def\maketitle{
  \null
  \thispagestyle{empty}
  % Cabecera ----------------
\begin{center}%
	{\noindent\Huge Universities of Burgos, León and Valladolid}\vspace{.5cm}%
	
	{\noindent\Large Master's degree}\vspace{.5cm}%
	
	{\noindent\Huge \textbf{Business Intelligence and Big Data in Cyber-Secure Environments}}\vspace{.5cm}%
\end{center}%

\begin{center}%
	\includegraphics[height=3cm]{img/escudoUBU} \hspace{1cm}
	\includegraphics[height=3cm]{img/escudoUVA} \hspace{1cm}
	\includegraphics[height=3cm]{img/escudoULE} \vspace{0.1cm}%
\end{center}%

  \vfill
  % Título proyecto y escudo informática ----------------
  \colorbox{cpardoBox}{%
    \begin{minipage}{0.9\textwidth}
      \vspace{.5cm}\large
      \begin{center}
      \textbf{Thesis of the Master's degree in Business Intelligence and Big Data in Cyber-Secure Environments}\vspace{.5cm}\\
      \textbf{\Large\@title{}}
      \end{center}
      \vspace{.2cm}
    \end{minipage}

  }%
  \hfill
  \vfill
  % Datos de alumno, curso y tutores ------------------
  \begin{center}%
  {%
    \noindent\Large
    Presented by \@author{}\\ 
    in University of Burgos --- \@date{}\\
    Supervisor: \@tutor{}\\
  }%
  \end{center}%
  \null
  \cleardoublepage
  }
\makeatother

\newcommand{\nombre}{Adrián Riesco Valbuena} %%% cambio de comando

% Datos de portada
\title{Extraction, transformation, loading and visualization of combined Twitter and Spotify data in a scalable architecture}
\author{\nombre}
\tutor{Álvar Arnaiz González}
\date{\today}

\begin{document}

\maketitle


\newpage\null\thispagestyle{empty}\newpage


%%%%%%%%%%%%%%%%%%%%%%%%%%%%%%%%%%%%%%%%%%%%%%%%%%%%%%%%%%%%%%%%%%%%%%%%%%%%%%%%%%%%%%%%
\thispagestyle{empty}


\noindent
\begin{center}%
	{\noindent\Huge Universities of Burgos, León and Valladolid}\vspace{.5cm}%
	
\begin{center}%
	\includegraphics[height=3cm]{img/escudoUBU} \hspace{1cm}
	\includegraphics[height=3cm]{img/escudoUVA} \hspace{1cm}
	\includegraphics[height=3cm]{img/escudoULE} \vspace{1cm}%
\end{center}%

	{\noindent\Large \textbf{Master's degree in Business Intelligence and Big Data in Cyber-Secure Environments}}\vspace{.5cm}%
\end{center}%



\noindent D. Álvar Arnaiz González, professor of the department named Computer Engineering, area named Computer Languages and Systems.

\noindent Exposes:

\noindent That the student Mr. \nombre, with DNI 71462231N, has completed the Thesis of the Master in Business Intelligence and Big Data in Cyber-Secure Environments titled ``Extraction, transformation, loading and visualization of combined Twitter and Spotify data in a scalable architecture''. 

\noindent And that thesis has been carried out by the student under the direction of the undersigned, by virtue of which its presentation and defense is authorized.
\begin{center} %\large
In Burgos, {\today}
\end{center}

\vfill\vfill\vfill

% Tutor
\begin{minipage}{0.45\textwidth}
\begin{flushleft} %\large
Approval of the Supervisor:\\[2cm]
D. Álvar Arnaiz González
\end{flushleft}
\end{minipage}


\newpage\null\thispagestyle{empty}\newpage

\frontmatter

% Abstract en castellano
\renewcommand*\abstractname{Resumen}
\begin{abstract}
Este proyecto consiste en el desarrollo de un proceso de Extracción, Transformación y Carga para capturar datos de las APIs de \textbf{Twitter} y \textbf{Spotify} APIs utilizando herramientas de software del ámbito del Big Data y con un ciclo de vida gestionado mediante metodologías ágiles. El proceso ETL consiste en la recolección de los tuits con el hashtag \texttt{\#NowPlaying} de la API de Twitter, la limpieza del texto y el aislamiento de los nombres de las canciones y artistas (eliminando caracteres fuera del alfabeto latino, hashtags, urls y menciones), la consulta a la API de Spotify para recoger los datos de las canciones, la combinación de ambos conjuntos de datos, su envío al Data Warehouse, su posterior consulta desde el back-end de la aplicación web y, finalmente, su presentación al usuario final en un front-end personalizado. Para servir los datos en el front-end se han diseñado dos visualizaciones, una consistente en una \textbf{tabla} que permite filtrar, ordenar y ocultar/mostrar columnas, y la otra consistente en una combinación de un formato de \textbf{barras} y con uno de \textbf{línea}, con un selector para cada uno de ellos, permitiendo al usuario ordenar por cualquier columna y cambiar la cantidad de datos mostrados. Las herramientas utilizadas son \textbf{Apache Airflow} como orquestador del flujo de datos, \textbf{Apache Spark} como ejecutor del proceso ETL, \textbf{Apache Cassandra} como almacén de datos, y una combinación de \textbf{Flask} y \textbf{Bootstrap}, junto con \textbf{Chart.js} y \textbf{Datables}, para la creación de la aplicación web personalizada. Todos los servicios se han encapsulado en contenedores dentro de la misma red utilizando \textbf{Docker Compose} como orquestador.
\end{abstract}

\renewcommand*\abstractname{Descriptores}
\begin{abstract}
Airflow, Apache, API, Big Data, Bootstrap, Cassandra, Chart.js, Datatables, Docker, Docker Compose, ETL, Flask, Spark, Spotify, Twitter.
\end{abstract}

\clearpage

% Abstract en inglés
\renewcommand*\abstractname{Abstract}
\begin{abstract}
This project consists of the development of an Extraction, Transformation and Loading process to capture data from \textbf{Twitter} and \textbf{Spotify} APIs using software tools from the Big Data domain and with a life cycle driven by agile methodologies. The ETL process consists of collecting from the Twitter API the tweets with the hashtag \texttt{\#NowPlaying}, cleaning the text and isolating the track and artist names (removing characters outside the Latin alphabet, hashtags, urls and mentions), querying the Spotify API to collect the track data, combining both data, sending it to the Data Warehouse, querying it from the back-end of the web application and, finally, serving it to the end user in a custom front-end. Two visualizations have been designed to serve the data in the front-end, one consisting of a \textbf{table} that allows filtering, sorting and hiding/showing columns, and the other consisting of a combination of a \textbf{bar} chart and a \textbf{line} chart, with a selector for each format and allowing the user to sort by any column and change the amount of data displayed. The tools used are \textbf{Apache Airflow} as flow orchestrator, \textbf{Apache Spark} as ETL process executor, \textbf{Apache Cassandra} as Data Warehouse, and a combination of \textbf{Flask} and \textbf{Bootstrap}, together with \textbf{Chart.js} and \textbf{Datatables}, to create the custom web application. All services have been encapsulated in containers within the same network using \textbf{Docker Compose} as the orchestrator.
\end{abstract}

\renewcommand*\abstractname{Keywords}
\begin{abstract}
Airflow, Apache, API, Big Data, Bootstrap, Cassandra, Chart.js, Datatables, Docker, Docker Compose, ETL, Flask, Spark, Spotify, Twitter.
\end{abstract}

\clearpage

% Indices
\tableofcontents

\clearpage

\listoffigures

\clearpage

\listoftables
\clearpage

\mainmatter

\addcontentsline{toc}{part}{Report}
\part*{Report}

\capitulo{1}{Introduction}

% Description of the work, the structure of the memory and the rest of the material delivered.

\nonzeroparskip Social networks are currently a fundamental aspect of society. People usually use social networks to share experiences, opinions, and aspects of their lives and interact with other people. Using social networks as a data source we can access a huge amount of information and be able to build accurate analyses on practically any topic.

\nonzeroparskip On the other hand, another aspect that has gradually permeated our society is the concept of subscriptions to services, be it music, movies, games, or almost any concept that we can think of. Not so many years ago, the concept of paying for subscriptions to services, where you do not actually own the content you pay for and instead get temporary access whose duration is defined by how long you continue to pay for the subscription, was relegated to very specific services and was not nearly as globalized as it is today.

\nonzeroparskip The global acceptance of subscription as a service is reflected in social networks, where users can comment on the different music, movie, and game platforms, turning each new release into a social phenomenon. In order to take advantage of both worlds, this project uses the social network Twitter to obtain information about the latest music listened to by users (by searching for a particular hashtag) and then consult the data of the song and the artist involved that Spotify, a platform based on music as a service, has.

\nonzeroparskip The development of the project has followed an agile methodology with two-week sprints, and has focused on learning and using recognized tools in the field of Big Data, such as Apache Airflow, Apache Spark, Apache Cassandra, Docker Compose and Flask, among others.
\capitulo{2}{Project objectives}

\nonzeroparskip The initial objectives through which the use case was built consisted in the following points:
\begin{itemize}
	\item Ability to obtain data in real time.
	\item Combination of at least two different data sources.
	\item Potential to scale in both technology and data volume.
	\item Involvement of various technologies in the Big Data field.
	\item Use of open source tools.
\end{itemize}

\nonzeroparskip After a research, the author designed the use case and built the project objectives:
\begin{itemize}
	\item Build a pipeline to gather information about last songs listened from the \textbf{Twitter API}.
	\item Find information about the songs (name, artist and audio features) through the \textbf{Spotify API}.
	\item Execute all the ETL\footnote{ETL is the acronym for Extract, Transform and Load, the three phases for data processing} process in \textbf{Apache Spark}.
	\item Store all the data in a Data Warehouse under a known technology, \textbf{Apache Cassandra}.
	\item Visualize the information in a custom front-end and back-end created with \textbf{Flask} and \textbf{Bootstrap}.
	\item Orchestrate all the data flow with \textbf{Apache Airflow}.
	\item Develop the project with \textbf{Docker} and \textbf{Docker Compose} to ensure deployment through heterogeneous environments.
\end{itemize}

\nonzeroparskip Through these global objectives, the low-level functional and technical requirements were specified, as shown in the appendix \ref{requirements}. The detailed use case is described in the section \ref{analysis}.
\capitulo{3}{Theoretical concepts}

%Theoretical concepts of \LaTeX \footnote{Example of footnote}.
%\subsection{Subsection}
%\subsubsection{Subsubsection}
%Use of cite \cite{wiki:latex}, \cite{koza92}.
%\imagen{img/escudoInfor}{Image caption}
%Lists
%\begin{itemize}
%Enumerate.
%\begin{enumerate}
%Description.
%\begin{description}

\nonzeroparskip In this section are covered the theoretical concepts in which the project has been based. All concepts are described in a detailed and simple way since this master's thesis can be aimed at technical and non-technical students.

\section{API}

\nonzeroparskip An Aplication Interface or API is an interface that defines the interactions that can be made with a software system. The APIs generally define the data that can be requested and sent to the system, the way to authenticate to it and the format of the returned data~\cite{ibm_restapi}.

\nonzeroparskip In relation to web development, most of the APIs work according to Hypertext Transfer Protocol (HTTP), a communication protocol that allows information transfers through files on the World Wide Web. Additionally and not exclusive, a large number of APIs are developed according to the REST architectural style, defined by Roy Fielding in the year 2000 and which is based on a series of principles that seek to facilitate development:
\begin{enumerate}
	\item Uniform interface for all resources, forcing all queries made to the same resource (each with a specific Uniform Resource Identifier or URI) to have the same form regardless of the origin of the request.
	\item Decoupling between the client and the server, making the only information that the client must know about the server is its identifier (URI) and that the only action to be carried out by the server is to return the data required in the request.
	\item Stateless queries, meaning that each request must contain all the information necessary to be processed without requiring an additional request or storing any type of state.
	\item Allow, whenever possible, both client-side and server-side caching to reduce the load of the former and increase the scalability of the latter.
	\item Layer system, allowing multiple intermediaries between the client and the server and preventing them from knowing in any case if they are communicating with the other party or with an intermediary.
	\item Although the resources exchanged are usually static, a REST architecture can optionally have responses that contain snippets of executable code.
\end{enumerate}

\nonzeroparskip In general terms, an API based on a REST architecture serves to make it easier for developers to develop applications that interact with the resources published by it.

\subsection{Twitter API}
\nonzeroparskip Twitter is an American social network founded in 2006 that allows users to share short posts (280 characters since 2017), known as tweets, and interact with those of other users through replies, likes, retweets or quotes~\cite{wikipedia_twitter}. Although it has recently incorporated additional payment functions, this social network is free to use and is accessible on multiple platforms. Currently, the social network has 217 million active users daily~\cite{variety_twitterusers}.

\nonzeroparskip On the other hand, Twitter is also known for giving certain facilities to developers to make their products interact with the platform. The company has a Twitter Developer portal where a multitude of resources and useful documentation are posted\cite{twitter_dev}. This portal contains a description of the API that Twitter offers, how to authenticate, the different endpoints to which queries can be launched, and the associated usage limits. 

\nonzeroparskip The Twitter API, currently in its second version, allows the user to request and receive a wide variety of data. Depending on the query launched, the user can receive a series of different objects, each with its own fields and parameters:
\begin{itemize}
	\item \textbf{Tweets}. It represents the basic block of communication between Twitter users.
	\item \textbf{Users}. It represents a user account and its metadata.
	\item \textbf{Spaces}. It represents a space (virtual places in Twitter where users can interact in live conversations) and its metadata.
	\item \textbf{Lists}. It represents a Twitter list (used to configure information visualized in the timeline) and its metadata.
	\item \textbf{Media}. It represents any image, video or GIF attached to a tweet and can be obtained by expanding the Tweet object.
	\item \textbf{Polls}. It represents a poll (choices, duration, end-time and results) and can be obtained by expanding the Tweet object.
	\item \textbf{Places}. It represents a place identified in a tweet and can be obtained by expanding the Tweet object.
\end{itemize}

\nonzeroparskip Of all the endpoints available in the API (manage tweets, user lookup, search spaces, full-archive tweet search...), in this thesis only the Recent Search endpoint has been used, which returns a list of the most recent tweets based on the rules entered in the query. Both the number of tweets to receive and the id or date of the oldest tweet to be returned can be specified, and this endpoint allows receiving up to one hundred tweets per query and includes a pagination token to handle larger results~\cite{twitter_dev_searchtweets}.

\subsection{Spotify API}
\nonzeroparskip Spotify is a company of Swedish origin founded in 2006 that provides audio streaming services, currently being one of the companies with the largest number of users among all those that have this type of service. Spotify has a catalog made up of music and podcasts, including more than 82 million songs, distributed through a free service (limited control and periodic announcements) with the option of a premium subscription. Its business model is based on advertising and paying users, and it pays royalties to artists based on the proportion of streaming of their songs compared to the total played~\cite{wikipedia_spotify}.

\nonzeroparskip Spotify has a developer portal that provides a wealth of documentation to help design and implement various use cases. Spotify has an API based on a REST architecture with different published endpoints that return metadata of artists, albums and songs from its own catalog, as well as information on users, lists and music saved by them, in JSON format\cite{spotify_dev}.

\nonzeroparskip The Spotify API has several endpoints to which queries can be sent to collect or modify information: Albums, Artists, Shows, Episodes, Search~\cite{spotify_dev_endpoint_searchforitem}, Tracks~\cite{spotify_dev_endpoint_gettracksaudiofeatures}, Users, Playlists, Categories, Genres, Player and Market. During this thesis the following have been used:
\begin{itemize}
	\item \textbf{Search}. Search for Item allows to obtain information about the Spotify catalog of artists, songs, albums, playlists, shows or episodes. You can specify the type or types of objects to return, in this case being the type ``tracks''.
	\item \textbf{Tracks}. Get Tracks' Audio Features allows to obtain the characteristics of a set of songs specified by their id. The characteristics returned are as follows:
	\begin{itemize}
		\item \textbf{Acousticness}. Confidence measure from 0.0 to 1.0 about whether the track is acoustic, with 1.0 representing high confidence that it is acoustic.
		\item \textbf{Danceability}. It describes with a value between 0.0 and 1.0 how suitable a track is for dancing based on a combination of its musical elements, with 1.0 being the greatest danceability.
		\item \textbf{Duration\_ms}. It represents the duration of the track in milliseconds.
		\item \textbf{Energy}. It represents with a value between 0.0 and 1.0 the conception of the energy level of the track, being 1.0 the maximum energy value.
		\item \textbf{Instrumentalness}. It predicts with a value between 0.0 and 1.0 whether or not the track contains vocals. Values above 0.5 usually represent tracks without vocals, and the closer to 1.0 the more likely they are.
		\item \textbf{Key}. It indicates the key the track is in, with each key having an assigned integer starting with 0. If no key is detected, the value is -1.
		\item \textbf{Liveness}. It represents audience presence with a value between 0.0 and 1.0. Values greater than 0.8 indicate a high probability that the track was recorded live.
		\item \textbf{Loudness}. Indicates the average volume of a track in decibels, with values generally contained between -60dB and 0dB.
		\item \textbf{Mode}. Indicates the modality of the track, being the value 1 greater and 0 less.
		\item \textbf{Speechiness}. Detects the presence of spoken words in a track. Values less than 0.33 typically indicate instrumental tracks without vocals, values between 0.33 and 0.66 songs with music and vocals, and values greater than 0.66 podcasts, audiobooks, and similar formats.
		\item \textbf{Tempo}. Indicates the tempo or rhythm of a track in beats per minute.
		\item \textbf{Time\_signature}. Represents the estimated time signature value, with values between 3 and 7 indicating 3/4 and 7/4 time signatures, respectively.
		\item \textbf{Valence}. Indicates with values between 0.0 and 1.0 the musical positivity transmitted by the track, where high values indicate greater positivity, while low values indicate greater negativity.
	\end{itemize}
\end{itemize}

\section{Orchestrator}

\nonzeroparskip Section explaining Flow Orchestrator -> Airflow.

\section{NoSQL Databases}

\nonzeroparskip Section explaining NoSQL Databases.

\section{Containers}

\nonzeroparskip Section explaining Containers.

\section{Continuous Integration / Continuous Delivery}

\nonzeroparskip Section explaining CI/CD.

\section{Template engines}

\nonzeroparskip Section explaining Template engines -> Jinja.

\section{Web Server Gateway Interface}

\nonzeroparskip Section explaining Web Server Gateway Interface (WSGI).


\section{Tables}

\nonzeroparskip TablaSmall.

\tablaSmall{Tools and technologies used}{l c c c c}{herramientasportipodeuso}
{ \multicolumn{1}{l}{Tools} & App AngularJS & API REST & BD & Memoria \\}{ 
HTML5 & X & & &\\
CSS3 & X & & &\\
BOOTSTRAP & X & & &\\
\TeX{}Maker & & & & X\\
Astah & & & & X\\
} 

\capitulo{4}{Techniques and tools}

\nonzeroparskip In this section are presented the methodological techniques and development tools used to carry out the project.

\section{GitHub}

\nonzeroparskip GitHub is the repository where the project was uploaded and its evolution was tracked.

\section{APIs}

\nonzeroparskip During this project there were used APIs from two different providers to gather the information: Twitter API and Spotify API.

\section{Postman}

\nonzeroparskip Postman is a tool that allows the user to build and use APIs in a simple way. Some of its characteristics are the API repository (easier storage, cataloging and collaboration), the availability of tools to help in the API design, testing and documentation, the workspaces to organize the work, and built-in integrations with tools such as GitHub, Azure DevOps, Jenkins, Splunk, Slack and Microsoft Teams. In addition, Postman is based on open source technologies, which provides the ability to be easily extended~\cite{postman}.

\section{Apache Airflow}

\nonzeroparskip Apache Airflow is a service orchestrator that allows you to plan, manage, and monitor workflows~\cite{airflow}. It was created in 2014 by Airbnb with the aim of handling the company's huge data flows, and published in 2015 under an open source license. In March 2016 the project joined the Apache Software Foundations incubator and was published as a top level project in 2019.

\nonzeroparskip Airflow is used to automate jobs by breaking them down into smaller tasks. For example, this project uses Airflow to automate the ETL process that consumes data from Twitter and Spotify, processes it, and serves it to the user. Among the main features of Airflow are scalability and ease of integration with other tools.

\nonzeroparskip The main element used by Airflow are the Directed Acyclic Graphs or DAGs, which are groups of tasks connected to each other through dependencies like the nodes of a graph. The word \textit{direct} indicates that the existing relationships in the graph must only have one direction (bidirectional relationships between nodes or tasks are not allowed), while the word \textit{acyclic} means that cycles cannot exist in the graph (nodes or tasks cannot be executed more than once). Tasks are defined by means of an operator and there is a very extensive library with operators that allow defining a wide variety of services such as BashOperator, to execute Bash commands, or SparkSubmitOperator, to submit a task to Spark. Regarding the programming language, Python is the one in which DAGs are developed.

\nonzeroparskip Airflow allows you to have control of the tasks executed through a record of their executions, the time, current or final status and the generated logs. In addition, it allows certain parameters to be associated with each task, such as, for example, the maximum execution time allowed.

\section{Apache Spark}

\nonzeroparskip Apache Spark is...

\section{Cassandra}

\nonzeroparskip Cassandra is a NoSQL database that...

\section{Flask}

\nonzeroparskip Flask is...

\section{Bootstrap}

\nonzeroparskip Bootstrap is...

\section{Docker}

\nonzeroparskip Docker is...

\section{Docker Compose}

\nonzeroparskip Docker Compose is...

\capitulo{5}{Relevant aspects of the project}

%This section aims to collect the most interesting aspects of the development of the project, commented by its authors.
%It must include from the exposition of the life cycle used, to the most relevant details of the analysis, design and implementation phases.
%It is sought that it is not a mere operation of copying and pasting diagrams and extracts from the source code, but that the solution paths that have been taken are really justified, especially those that are not trivial.
%It may be the most appropriate place to document the most interesting aspects of the design and implementation, with a greater emphasis on aspects such as the type of architecture chosen, the indexes of the database tables, normalization and denormalization, distribution in files3, business rules within databases (EDVHV GH GDWRV DFWLYDV), development aspects related to the WWW...
%This section must become the summary of the practical experience of the project, and by itself justifies that the report becomes a useful document, a reference source for authors, tutors and future students.

\nonzeroparskip The first step of the project was the feasibility and viability analysis of the concept devised. The author was looking to use two data sources with:

\begin{itemize}
	\item Real and updated data, preferable related to the social interest.
	\item The possibility of getting a stream data flow.
	\item The potential to combine both to get an added value.
\end{itemize}

\nonzeroparskip Considering the previous points, the author found an interesting option on Twitter and Spotify providers. Both of them provides solid APIs for a fluid development and have the characteristics needed to combine the data collected. Consequently, the author designed the following use case:

\begin{enumerate}
	\item The Twitter API is consulted to gather the \textit{tweets} with the hashtag \textit{\#NowPlaying}.
	\item The tweet is cleaned, removing the stopwords and the other hashtags and getting the song name and artist as isolated as possible.
	\item The Spotify API is consulted to gather the information of the song identified.
	\item The vector values of the cleaned Twitter data and the name of the song returned by Spotify are compared to ensure they are the same.
	\item The data is moved to the database, ready to be stored and visualized.
\end{enumerate}

\nonzeroparskip During the design phase, the author analyzed the output of both APIs using Postman. In the first place, relying on the Twitter documentation, the author inspected the Twitter API by following the next steps:
\begin{enumerate}
	\item Get access to the Twitter Developer Portal.
	\item Get the credentials needed to consult the different endpoints of the API.
	\item Import the \textit{Twitter API v2} collection on Postman.
	\item Modify the automatically created environment called \textit{Twitter API v2} to include the following developer keys and tokens:
	\begin{itemize}
		\item Consumer key (consumer\_key).
		\item Consumer secret (consumer\_secret).
		\item Access token (access\_token).
		\item Token secret (token\_secret).
		\item Bearer token (bearer\_token).
	\end{itemize}
	\item In the collection tab, select the endpoint \textit{Tweet Lookup} -> \textit{Single Tweet} for the initial exploration.
	\item Create a fork and configure the following parameters:
	\begin{itemize}
		\item[tweet.fields] 
		\item[expansions]
		\item[id] - The identifier of a tweet similar to the ones to gather. It can be find in the tweet's URL.
		\item Token secret (token\_secret).
		\item Bearer token (bearer\_token).
	\end{itemize}
	\item Get access to the Twitter Developer Portal.
\end{enumerate}


\nonzeroparskip The project development was undertaken following an Agile methodology.\\
\capitulo{6}{Related works}

\nonzeroparskip Both the software tools used in the project, widely recognized within the Big Data industry, and the data sources, from large cap and renowned companies, have been applied to countless projects throughout their lifetime.

\nonzeroparskip During the planning and study phase of the APIs and tools, the author discovered several projects that have combined both elements in different ways. Although no project has been found that combines both APIs and may involve a data flow similar to that of the present project, the author has decided to mention those that have been considered most interesting either for their data flow or for the originality of the idea:

\begin{itemize}
	\item \textbf{Tweetpy}~\footnote{\url{https://www.tweepy.org/}}. Created in 2009 by Joshua Roesslein, this project contributes a Python library to interact with the Twitter API, simplifying interactions during development. Despite considering it as an alternative, the author decided not to make use of it in order to learn directly how to interact with the API. However, in the case of having to interact with user accesses, it would probably have been chosen.
	\item \textbf{Spotipy}~\footnote{\url{https://spotipy.readthedocs.io/}}. Created in 2014 by Paul Lamere, this project provides a Python library to interact with the Spotify API. As with the previous library, it was decided not to use it to learn from direct interaction with the API, since the data flow of the project does not involve complex access permissions.
	\item \textbf{Divide for Spotify}~\footnote{\url{https://divideforspotify.com/}}. This project makes use of the Spotify API to divide the songs that have been liked among the playlists of the user in question. This project has been chosen for its originality, as it takes advantage of the API to add a very interesting functionality to the user of the platform.
	\item \textbf{Spotify to Twitter}~\footnote{\url{https://github.com/transitive-bullshit/spotify-to-twitter}}. This project combines both APIs to automate a Twitter account to publish songs from a Spotify playlist.
	\item \textbf{}~\footnote{\url{}}.  **PENDIENTE COMPLETAR**
\end{itemize}

Trabajos similares con conjuntos de datos o flujo de datos similar. Hashtags, músicas... Herramientas o webs que hagan esos. 2 o 3 herramientas. Análisis de Twitter. Twitter trending topic analysis. Pueden ser artículos científicos.
\capitulo{7}{Conclusions and future work lines}

% Every project must include the conclusions derived from its development. These can be of a different nature, depending on the type of project, but normally there will be a set of conclusions related to the results of the project and a set of technical conclusions.
% In addition, it is very useful to make a critical report indicating how the project can be improved, or how work can continue along the lines of the completed project.

% Next steps: clean data, check songs, ask for last id tweet instead of get last n (if no rate limits), add group by, add global metrics, add range of metrics, group by artist, optimize the process (duration), button to load from historic...
% Create MV, developer keys, change GitHub name, change project name, video demo, add summary, descriptors, add figures.

% Alvar:
% - ¿Eliminar resumen y descriptores en español?
% - ¿Título del TFM y del GitHub?
% Twitter and Spotify data extraction, transformation, loading and visualization using Big Data and development tools.
% - ¿Claves de desarrollador para la MV?

% - Más problemas? Comprensión inicial de Docker? Configuración inicial de Airflow?
% - Capturas de UIs -> Explicación breve de los elementos de cada vista.
% - Captura de Aiflow
% - Agradecimientos.
% - ¿?

\nonzeroparskip Given the requirements established in the planning phase, the final result is considered a success. The author has learned how to handle the tools included in the scope and how to interconnect them to achieve a complete ETL flow with a simple but effective visualization layer.

\nonzeroparskip In case of continuing with the presented work, the author has performed an exercise of analysis and identification of the next steps and improvements that could be implemented:
\begin{itemize}
	\item \textbf{Improve the percentage of correctly identified tracks.} This action could be addressed in several ways, including: discarding tweets with characters outside the Latin alphabet (not only specific characters), removing stop words in several languages so as not to have to discard all characters not included in the Latin alphabet, or increasing the number of tracks returned by Spotify and performing a comparative method (bag of words, TI-IDF, etc.) between their titles and the full text of the tweet. This would improve the reliability of the database and give more confidence when building the global metrics.
	\item \textbf{Ensure that we capture as much data as possible.} In the case of wanting to develop the project on a larger scale, both providers can be contacted to try to obtain an increase in the consumption limit of their APIs. In this way, depending on the new limitation, it could try to divide its catch at a certain time. In the best case, it would be possible to capture all tweets, even if our system captures them periodically, since the identifier of the last tweet captured in the previous iteration can be used as a reference for the endpoint to start the new capture from that point.
	\item \textbf{Add visualizations in the front-end layer.} Additional visualizations could be introduced that would allow the user, for example, to group by artist and to obtain global metrics that answer questions such as: ``How popular are this artist's tracks?'' or ``How danceable are this artist's tracks compared to this other artist's tracks?''. Another possible view to develop would be one that allows tracks to be observed by metric ranges and answers questions such as what percentage of the total tracks are between 80-100 of the energy metric.
	\item \textbf{Upload from history functionality.} Another improvement that would facilitate user interaction with the tool would be the ability to load the file history from the tool itself. In this way, the user could stop the process, select his historical files and be able to load only the desired data.
	\item \textbf{Replace Docker Compose with a more suitable tool.} At the design level, although Docker Compose fits the requirements for this project, in the case that a large scalability is required, it would be convenient to replace the tool with another one that allows container management across multiple hosts, such as Docker Swarm or Kubernetes.
	\item \textbf{Refactor the code.} In the event that the process increases in the amount of data captured, it is advisable to perform an analysis of the capture process and a refactoring of the necessary parts to speed up the process, if necessary. With the current data volume, the execution speed of the whole process triggered by Apache Airflow takes an average of 30-35 seconds.
	\item \textbf{Partial or total migration to the cloud.} Finally, if deemed consistent with new requirements that may arise in the future, parts or even the project could be migrated to a cloud infrastructure. For example, the Data Warehouse could be replaced by one of the native offerings of the major Cloud providers.
\end{itemize}

\nonzeroparskip Most of these steps were considered for inclusion at some point in the development phase, but were discarded to ensure that the project, which for time reasons was conceived from the beginning as a proof-of-concept rather than an end-user oriented application, remained on schedule.

% Añadir entrada en el índice: Anexos
\appendix
\addcontentsline{toc}{part}{Appendixes}
\part*{Appendixes}

\apendice{Project Plan}

\section{Introduction}
\nonzeroparskip The project planning was decided in an initial meeting between the author and its tutor. It was based in an Agile methodology, with two-weeks \textit{sprints} and meetings between the author and his tutor conditioned to their availability.\\

\nonzeroparskip The project repository was stored in GitHub under the url \url{https://github.com/AdrianRiesco/Data-Engineer-project}. Each \textit{sprint} was created as an \textit{milestone}, with the \textit{issues} contained there being the tasks asigned. The \textit{issues} were created to reflect tasks at most eight hours, allowing the author segregate his work and manage each \textit{sprint} better. The author closed an \textit{issue} when the task was finished and a \textit{milestone} when the \textit{sprint} was over, regardless of its state. If a task remained in an open state when a \textit{sprint} reached its planned end date, the \textit{issue} was transfered to the next \textit{milestone}.\\

\nonzeroparskip A meeting was held by the author and his tutor at the end of each sprint. During these meetings, both of them reviewed the state and development of the tasks of the corresponding sprint and planned the tasks of the next sprint. All the \textit{milestones} and \textit{issues} can be consulted in the project repository.\\

\section{Temporary planning}
\nonzeroparskip The sprints carried out for the development of the project are described below with they correspondant dates:
\begin{description}
	\item[Initial meeting.] Held on Monday January 31st, it was the start point for the first sprint. During this meeting, the objective of the project, the data source and the tools to be used were validated by both the author and his tutor. The author previously made a research and came with an idea and the tutor exposed his point of view to create the final goal.
	\item[Sprint 1.] Weeks of January 31st and February 7th. This Sprint had the following tasks assigned:
	\begin{itemize}
		\item Configure the work environment.
		\item Configure the project memory template.
		\item Write a draft of the objectives and main goals.
		\item Write a brief description of the tools selected.
		\item Write a brief explanation of the selected tools and the work methodology.
		\item Inspect Twitter API.
		\item Inspect Spotify API.
	\end{itemize}
	The end-of-sprint meeting was held on Wednesday February 16th.
	Analysis: Most of the activities were realized by the author, excepting the Inspection of the Spotify API. Regarding the Twitter API, the author inspected the output and he concluded that it had the characteristics needed to be used to launch queries to the Spotify API (the tweet could be cleaned to get the song name and artist).
	\item[Sprint 2.] Weeks of February 14th and February 21st. This Sprint had the following tasks assigned:
	\begin{itemize}
		\item Task1.
	\end{itemize}
	The end-of-sprint meeting was held on M--- February --th.
	\item[Sprint 3]. Weeks of February 28th and March 7th. This Sprint had the following tasks assigned:
	\begin{itemize}
		\item Task1.
	\end{itemize}
	The end-of-sprint meeting was held on M--- March --th.
	\item[Sprint 4]. Weeks of March 14th and March 21st. This Sprint had the following tasks assigned:
	\begin{itemize}
		\item Task1.
	\end{itemize}
	The end-of-sprint meeting was held on M--- March --th.
	\item[Sprint 5]. Weeks of March 28th and April 4th. This Sprint had the following tasks assigned:
	\begin{itemize}
		\item Task1.
	\end{itemize}
	The end-of-sprint meeting was held on M--- April --th.
	\item[Sprint 6]. Weeks of April 11th and April 18th. This Sprint had the following tasks assigned:
	\begin{itemize}
		\item Task1.
	\end{itemize}
	The end-of-sprint meeting was held on M--- April --th.
	\item[Sprint 7]. Weeks of April 25th and May 2nd. This Sprint had the following tasks assigned:
	\begin{itemize}
		\item Task1.
	\end{itemize}
	The end-of-sprint meeting was held on M--- May --th.
	\item[Sprint 8]. Weeks of May 9th and May 16th. This Sprint had the following tasks assigned:
	\begin{itemize}
		\item Task1.
	\end{itemize}
	The end-of-sprint meeting was held on M--- May --th.
	\item[Sprint 9]. Weeks of May 23rd and May 30th. This Sprint had the following tasks assigned:
	\begin{itemize}
		\item Task1.
	\end{itemize}
	The end-of-sprint meeting was held on M--- June --th.
	\item[Sprint 10]. Weeks of June 6th and May 13th. This Sprint had the following tasks assigned:
	\begin{itemize}
		\item Task1.
	\end{itemize}
	The end-of-sprint meeting was held on M--- June --th.
\end{description}

\section{Feasibility study}
\nonzeroparskip The architecture of the project and the use case were designed to ensure its feasibility.

\subsection{Economic feasibility}
\nonzeroparskip The project is based on open-source platforms to ensure its economic and legal feasibility. The APIs where the information was gathered are free to use if the developer keeps his queries under specific limit rates.

\subsection{Legal feasibility}
\nonzeroparskip The project is based on open-source platforms to ensure its economic and legal feasibility. 


\apendice{Requirements} \label{requirements}

\section{Introduction}

\nonzeroparskip This section lists the general objectives and requirements identified during the initial planning of the project and on whose fulfillment the development of the project has focused.

\section{General objectives}

\nonzeroparskip The requirements through which the use case was built were the following:
\begin{itemize}
	\item Ability to obtain data in real time.
	\item Combination of at least two different data sources.
	\item Potential to scale in both technology and data volume.
	\item Involvement of various technologies in the Big Data field.
	\item Use of open source tools.
\end{itemize}

\section{Catalog of requirements}

\subsection{Functional requirements}

\nonzeroparskip The functional requirements (FR) that the project had to meet were:
\begin{itemize}
	\item \textbf{FR1}. The data must be obtained from the Twitter hashtag \texttt{\textit{\#NowPlaying}} every 30 minutes, taking care of API rate limits.
	\begin{itemize}
		\item \textbf{Status}: Met.
		\item \textbf{Rationale}: The Apache Airflow DAG was configured with a run frequency of 30 minutes. The script used to collect the data has a maximum of 100 tweets collected per run to comply with API rate limits.
	\end{itemize}
	\item \textbf{FR2}. There must be at least two different visualizations and one of them must provide the ability to view all of the stored data.
	\begin{itemize}
		\item \textbf{Status}: Met.
		\item \textbf{Rationale}: There are two visualizations in the web application, ``Data'' and ``Visuals'', and the first one displays all the stored data.
	\end{itemize}
	\item \textbf{FR3}. At least one of the visualizations must show last songs name, artist and audio features.
	\begin{itemize}
		\item \textbf{Status}: Met.
		\item \textbf{Rationale}: The ``Data'' view of the web application displays all required fields.
	\end{itemize}
	\item \textbf{FR4}. At least one of the visualizations must have a link to the source tweet.
	\begin{itemize}
		\item \textbf{Status}: Met.
		\item \textbf{Rationale}: The ``Data'' view of the web application contains a link to the source tweet.
	\end{itemize}
	\item \textbf{FR5}. At least one of the visualizations must have the ability to compare different metrics.
	\begin{itemize}
		\item \textbf{Status}: Met.
		\item \textbf{Rationale}: The ``Visuals'' view of the web application provides the ability to visually compare metrics.
	\end{itemize}
	\item \textbf{FR6}. At least one of the visualizations must combine two different types of visualizations.
	\begin{itemize}
		\item \textbf{Status}: Met.
		\item \textbf{Rationale}: The ``Visuals'' view of the web application combines bar and line formats in the same chart.
	\end{itemize}
	\item \textbf{FR7}. Both visualizations must provide sorting capabilities.
	\begin{itemize}
		\item \textbf{Status}: Met.
		\item \textbf{Rationale}: Both ``Data'' and ``Visuals'' views of the web application provides provides the ability to sort by metric.
	\end{itemize}
	\item \textbf{FR8}. Both visualizations must be responsive to different screen sizes.
	\begin{itemize}
		\item \textbf{Status}: Met.
		\item \textbf{Rationale}: The web application was developed using Bootstrap as the CSS framework to facilitate scaling and resizing on different screens.
	\end{itemize}
\end{itemize}

\subsection{Technical requirements}

\nonzeroparskip The technical requirements (TR) that the project had to meet were:
\begin{itemize}
	\item \textbf{TR1}. The development must have the ability to be deployed in different environments with minimum effort.
	\begin{itemize}
		\item \textbf{Status}: Met.
		\item \textbf{Rationale}: The services were developed using multiple containers managed via Docker Compose.
	\end{itemize}
	\item \textbf{TR2}. The data flow must be automated, with the entire process orchestrated by a single tool.
	\begin{itemize}
		\item \textbf{Status}: Met.
		\item \textbf{Rationale}: All data flow is orchestrated by Apache Airflow.
	\end{itemize}
	\item \textbf{TR3}. The execution of the ETL process must be done with a tool that can scale and run in distributed environments.
	\begin{itemize}
		\item \textbf{Status}: Met.
		\item \textbf{Rationale}: Data extraction, transformation and loading (into a .csv file) is processed by Apache Spark, and data loading to Cassandra is performed by a Cassandra Query Language shell (cqlsh) command.
	\end{itemize}
	\item \textbf{TR4}. The data warehouse must have the ability to escalate in terms of a Big Data problem.
	\begin{itemize}
		\item \textbf{Status}: Met.
		\item \textbf{Rationale}: Apache Cassandra is the Data Warehouse used.
	\end{itemize}
	\item \textbf{TR5}. The web application must be designed with widely recognized tools.
	\begin{itemize}
		\item \textbf{Status}: Met.
		\item \textbf{Rationale}: The web application was designed with Flask and Bootstrap, and the visualizations with Chart.js and Datatables.
	\end{itemize}
	\item \textbf{TR6}. All the tools used must be open source.
	\begin{itemize}
		\item \textbf{Status}: Met.
		\item \textbf{Rationale}: All the tools used to develop the project are open source.
	\end{itemize}
\end{itemize}
\apendice{Design specification}

\section{Introduction}

\section{Data design}

\subsection{Twitter data structure}
\nonzeroparskip The data received from Twitter queries has the following structure (the data has been obtained from a real query and its output has been reduced by trimming certain elements due to their length):

\begin{lstlisting}[language=json,firstnumber=1]
{"data":[
      {
         "id":"1533311938209382403",
         "entities":{
            "annotations":[
               {
                  "start":43,
                  "end":62,
                  "probability":0.6061,
                  "type":"Other",
                  "normalized_text":"Breakfast In America"
               }
            ],
            "urls":[
               {
                  "start":84,
                  "end":107,
                  "url":"https://t.co/YiXxSepm8x",
                  "expanded_url":"https://rideshare.airtime.pro",
                  "display_url":"rideshare.airtime.pro",
                  "images":[
                     {
                        "url":"https://pbs.twimg.com/news_img/1532561096858640397/3mUiSDDN?format=jpg&name=orig",
                        "width":1920,
                        "height":1200
                     },
                     {
                        "url":"https://pbs.twimg.com/news_img/1532561096858640397/3mUiSDDN?format=jpg&name=150x150",
                        "width":150,
                        "height":150
                     }
                  ],
                  "status":200,
                  "title":"Rideshare Radio",
                  "description":"Hits from the 70's 80s 90s 00s 10s 20s, No Talking just back to back Music totally commercials free 24/7",
                  "unwound_url":"https://rideshare.airtime.pro"
               }
            ],
            "hashtags":[
               {
                  "start":0,
                  "end":11,
                  "tag":"Nowplaying"
               },
               {
                  "start":129,
                  "end":139,
                  "tag":"Rideshare"
               }
            ]
         },
         "created_at":"2022-06-05T04:57:08.000Z",
         "text":"#Nowplaying 2010 Remastered - Supertramp - Breakfast In America  - Stream here-&gt; https://t.co/YiXxSepm8x - Non Stop Hits 24/7 #Rideshare #Radio #Hits #Uber #RideshareRadio #Petrol #Parcoursup #PlatinumJubilee #ENGvNZ #PrideMonth"
      },
      {
         Second tweet.      
      }
   ],
   "meta":{
      "newest_id":"1533311938209382403",
      "oldest_id":"1533311921256124416",
      "result_count":10,
      "next_token":"b26v89c19zqg8o3fpyzltxkhapj3hf1q96mc01w4yl3el"
   }
}
\end{lstlisting}

\nonzeroparskip The fields required for the project are:
\begin{itemize}
	\item \textbf{id.} Tweet id, useful uniquely identify the tweet.
	\item \textbf{text.} Tweet text, useful to identify the song played.
	\item \textbf{entities.} useful to clean the text and remove hashtasg, cashtags, mentions and urls.
	\item \textbf{created\_at.} Tweet creation date.
\end{itemize}

\subsection{Spotify data structure}
\nonzeroparskip The data received from Spotify queries to Search endpoint (``Search for Item'') has the following structure (the data has been obtained from a real query and its output has been reduced by trimming certain elements due to their length):
\begin{lstlisting}[language=json,firstnumber=1]
{"tracks":{
      "href":"https://api.spotify.com/v1/search?query=Savoy+Brown+Wang+Dang+Doodle+&type=track&offset=0&limit=1",
      "items":[
         {
            "album":{
               "album_type":"album",
               "artists":[
                  {
                     "external_urls":{
                        "spotify":"https://open.spotify.com/artist/17obwOahRWI121iMUZznh2"
                     },
                     "href":"https://api.spotify.com/v1/artists/17obwOahRWI121iMUZznh2",
                     "id":"17obwOahRWI121iMUZznh2",
                     "name":"Savoy Brown",
                     "type":"artist",
                     "uri":"spotify:artist:17obwOahRWI121iMUZznh2"
                  }
               ],
               "available_markets":[
                  "MX",
                  "US"
               ],
               "external_urls":{
                  "spotify":"https://open.spotify.com/album/5oq20r8iNOO9fpw8R2h3vE"
               },
               "href":"https://api.spotify.com/v1/albums/5oq20r8iNOO9fpw8R2h3vE",
               "id":"5oq20r8iNOO9fpw8R2h3vE",
               "images":[
                  {
                     "height":640,
                     "url":"https://i.scdn.co/image/ab67616d0000b2735dd44bf0a252e30d4bb2e7c8",
                     "width":640
                  },
                  {
                     "height":300,
                     "url":"https://i.scdn.co/image/ab67616d00001e025dd44bf0a252e30d4bb2e7c8",
                     "width":300
                  }
               ],
               "name":"Street Corner Talking",
               "release_date":"1971-01-01",
               "release_date_precision":"day",
               "total_tracks":8,
               "type":"album",
               "uri":"spotify:album:5oq20r8iNOO9fpw8R2h3vE"
            },
            "artists":[
               {
                  "external_urls":{
                     "spotify":"https://open.spotify.com/artist/17obwOahRWI121iMUZznh2"
                  },
                  "href":"https://api.spotify.com/v1/artists/17obwOahRWI121iMUZznh2",
                  "id":"17obwOahRWI121iMUZznh2",
                  "name":"Savoy Brown",
                  "type":"artist",
                  "uri":"spotify:artist:17obwOahRWI121iMUZznh2"
               }
            ],
            "available_markets":[
               "MX",
               "US"
            ],
            "disc_number":1,
            "duration_ms":440733,
            "explicit":false,
            "external_ids":{
               "isrc":"GBF077120720"
            },
            "external_urls":{
               "spotify":"https://open.spotify.com/track/7p99XDR7dKaIMTYV3zia0V"
            },
            "href":"https://api.spotify.com/v1/tracks/7p99XDR7dKaIMTYV3zia0V",
            "id":"7p99XDR7dKaIMTYV3zia0V",
            "is_local":false,
            "name":"Wang Dang Doodle",
            "popularity":19,
            "preview_url":"None",
            "track_number":7,
            "type":"track",
            "uri":"spotify:track:7p99XDR7dKaIMTYV3zia0V"
         }
      ],
      "limit":1,
      "next":"https://api.spotify.com/v1/search?query=Savoy+Brown+Wang+Dang+Doodle+&type=track&offset=1&limit=1",
      "offset":0,
      "previous":"None",
      "total":11
   }
}
\end{lstlisting}

\nonzeroparskip The fields required for the project are:
\begin{itemize}
	\item \textbf{id.} Track id, useful uniquely identify the track.
	\item \textbf{name.} Track name, useful to identify the track.
	\item \textbf{popularity.} Popularity of the track.
	\item \textbf{artists' id.} ID of the artists.
	\item \textbf{artists' name.} Name of the artists.
\end{itemize}

\nonzeroparskip The data received from Spotify queries to the Tracks endpoint (``Get Tracks' Audio Features'') has the following structure (the data has been obtained from a real query and its output has been reduced by trimming certain elements due to their length):
\begin{lstlisting}[language=json,firstnumber=1]
{"audio_features":[
      {
         "danceability":0.516,
         "energy":0.36,
         "key":7,
         "loudness":-11.264,
         "mode":1,
         "speechiness":0.03,
         "acousticness":0.83,
         "instrumentalness":0.885,
         "liveness":0.116,
         "valence":0.144,
         "tempo":127.176,
         "type":"audio_features",
         "id":"7dg3XqARw7qOrkt9pZZNRF",
         "uri":"spotify:track:7dg3XqARw7qOrkt9pZZNRF",
         "track_href":"https://api.spotify.com/v1/tracks/7dg3XqARw7qOrkt9pZZNRF",
         "analysis_url":"https://api.spotify.com/v1/audio-analysis/7dg3XqARw7qOrkt9pZZNRF",
         "duration_ms":233812,
         "time_signature":4
      },
      {
         Group of features of the second track.
      }...]
}
\end{lstlisting}

\nonzeroparskip The fields required for the project are:
\begin{itemize}
	\item id.
	\item danceability.
	\item energy.
	\item key.
	\item loudness.
	\item mode.
	\item speechiness.
	\item acousticness.
	\item instrumentalness.
	\item liveness.
	\item valence.
	\item tempo.
	\item duration\_ms.
	\item time\_signature.
\end{itemize}

\subsection{Cleaned data}
\nonzeroparskip After the cleaning process, the resulting data structure is:

\begin{itemize}
	\item id\_tweet.
	\item text.
	\item created\_at.
	\item url\_tweet.
	\item id\_track.
	\item name.
	\item popularity.
	\item artists\_id.
	\item artists\_name.
	\item danceability
	\item energy
	\item key
	\item loudness
	\item mode
	\item speechiness
	\item acousticness
	\item instrumentalness
	\item liveness
	\item valence
	\item tempo
	\item duration\_ms
	\item time\_signature
\end{itemize}

\section{Procedural design}
\nonzeroparskip Flow diagram.

\section{Architectural design}
\nonzeroparskip Component diagram.
\apendice{Programming technical documentation}

\section{Introduction}

\nonzeroparskip This section contains the directory structure used and the main technical details that a user wishing to reproduce or execute the project should be aware of.

\section{Directory structure}
\nonzeroparskip The project repository, hosted on GitHub, has the following directory structure:
\begin{itemize}
	\item \textbf{\texttt{airflow}}. It contains the created DAG and the folders that Airflow needs to function properly.
	\item \textbf{\texttt{cassandra}}. I contains the database schema required.
	\item \textbf{\texttt{doc}}. It contains the project report and the \LaTeX files used to generate it.
	\item \textbf{\texttt{docker}}. It contains the Docker Compose file required to deploy the project and the Airflow, Spark and Flask folders used for building custom images.
	\item \textbf{\texttt{spark}}. It contains the script used to collect the API data, the necessary libraries and the history files.
\end{itemize}

\nonzeroparskip The directory structure is shown in the figure~\ref{directory}. \figuraNormalSinMarco{0.40}{img/directory}{Directory structure}{directory}{}

\section{Programmer's guide}

\subsection{Analysis}

\nonzeroparskip During the analysis phase, the author inspected the output of Twitter and Spotify APIs using Postman. In the first place, relying on the Twitter documentation, the author inspected the Twitter API by following the next steps:
\begin{enumerate}
	\item Get access to the Twitter Developer Portal.
	\item Get the credentials needed to consult the different endpoints of the API.
	\item Import the \textit{Twitter API v2} collection on Postman.
	\item Create a fork of the automatically created environment (\textit{Twitter API v2}) and collection \textit{Twitter API v2} to be able to edit the values.
	\item Modify the environment to include the following developer keys and tokens:
	\begin{itemize}
		\item Consumer key (\texttt{consumer\_key}).
		\item Consumer secret (\texttt{consumer\_secret}).
		\item Access token (\texttt{access\_token}).
		\item Token secret (\texttt{token\_secret}).
		\item Bearer token (\texttt{bearer\_token}).
	\end{itemize}
	\item In the collection tab, select the endpoint \textit{Search Tweets} $\longrightarrow$ \textit{Recent search} for the initial exploration. Configure the following parameters:
	\begin{itemize}
		\item \texttt{query} = \texttt{\#NowPlaying}
		\item \texttt{tweet.fields} = \texttt{created\_at,entities}
		\item \texttt{max\_results} = \texttt{10}
	\end{itemize}
	\item Now, the query can be sent \url{https://api.twitter.com/2/tweets/search/re cent?query=\%23NowPlaying\&max\_results=10\&tweet.fields=created\_a t,entities} to get the 10 most recent tweets with the hashtag \texttt{\textit{\#NowPlaying}} and receive their basic information (id, text) as well as the entities (hashtags, urls, annotations...) and the creation time stamps.
\end{enumerate}

\nonzeroparskip After analyze the data gathered from the Twitter API,  the author inspected the Spotify API (more specifically the endpoint ``Search for Item''), following the next steps:
\begin{enumerate}
	\item Get access to the Spotify Developer Portal.
	\item Enter in the developers console and select the ``Search for Item'' endpoint.
	\item Specify the parameters of the search. We can specify the type ``track'' and add a limit of one to only receive the first song found.
	\item After click on get the bearer token, we can use that token by clicking on Try Me or just copy the resulting query in our Linux console to check the output.
	\item The result of this query is the first result of the search containing information of the artist, the song and the album. With the artist and song ids we can consult other endpoints to get an audio analysis, the audio features and the artist information, between others.
\end{enumerate}

\subsection{Development} \label{programmer_development}
\nonzeroparskip To run the project, the following items must be installed in the system:
\begin{itemize}
	\item Docker\footnote{\url{https://docs.docker.com/engine/install/ubuntu/}}. The author used the version \texttt{20.10.12}, build \texttt{e91ed57}.
	\item Docker Compose\footnote{\url{https://docs.docker.com/compose/install/}}. The author used the version \texttt{1.29.2}, build \texttt{5becea4c}.
	\item Python\footnote{\url{https://docs.python-guide.org/starting/install3/linux/}}. The author used the version \texttt{3.8.10}.
\end{itemize}

\nonzeroparskip The above packaged were installed in a ``\texttt{Ubuntu 20.04.4 LTS}'' virtual machine. In case of using a different operating system, please refer to the package documentation.

\nonzeroparskip It is important to note that to run or play the project, the user has to modify the \texttt{.env} file located within the \texttt{docker} folder and include the respective developer keys for both Twitter and Spotify.

\section{Compilation, installation and execution of the project} \label{programmer_execute}
\nonzeroparskip The steps needed to run the project are:
\begin{enumerate}
	\item Ensure that the folders \texttt{spark/resources/history}, \texttt{airflow/logs} and \texttt{airflow/plugins} have the necessary permissions by executing the command \texttt{sudo chmod -R 777 folder\_to\_set\_permissions}. 
	\item Open a \texttt{command prompt} and move to the \texttt{docker} folder.
	\item Launch the environment with ``\texttt{sudo docker-compose up}''. It can be launched in the background by adding the flag \texttt{-b}: ``\texttt{sudo docker-compose up -b}''.
	\item Wait until all the services have started and are in a healthy state. It usually is achieved when the \texttt{airflow\_webserver\_container} is continuously displaying a status message, if the flag \texttt{-b} was not included. Otherwise, a list of the services can be displayed with the command ``\texttt{sudo docker ps}''.
	\item Access to the Airflow UI (\url{http://localhost:8080}, user ``\texttt{airflow}'' and password ``\texttt{airflow}'') and start the DAG ``\texttt{spark\_main}'' as shown in the figure~\ref{start-dag}.
	\item Access to the web application (\url{http://localhost:8000}) and check that the ``\texttt{Data}'' and ``\texttt{Visuals}'' views contain the captured data.
	\item To stop the containers, the command ``\texttt{sudo docker-compose down}'' can be used. To remove the networks and docker volumes, ``\texttt{sudo docker-compose down -v}'' must be used. The process can also be stopped by pausing the DAG as explained in step 5.
\end{enumerate}

\figuraNormalSinMarco{0.25}{img/start-dag}{Pause/unpause DAG}{start-dag}{}

\nonzeroparskip The author faced some problems during the development phase that are easily solved:
\begin{itemize}
	\item If the build fails for an error in the Airflow container related with the \texttt{/opt/airflow/logs} folder, it will probably be due to the permissions in the \texttt{airflow/logs} folder. To solve it, open a command prompt in the \texttt{airflow} folder and execute the commands ``\texttt{sudo chmod -R 777 logs}'' and ``\texttt{sudo chmod -R 777 plugins}''.
	\item If the DAG fails, the logs can be consulted in the \texttt{airflow/logs/dag\_id =spark\_main} folder. If the ``\texttt{cassandra\_load}'' task is failing, it can be due to the permissions of the \texttt{spark/resources/history} folder. To solve that, open a command prompt in the \texttt{spark/resources} folder and execute the command ``\texttt{sudo chmod -R 777 history}''. Then, remove the ``\texttt{data.csv}'' file located in the \texttt{spark/resources} folder (or manually create the history file) and execute the DAG again.
\end{itemize}

\section{System tests}
\nonzeroparskip To test that the project is working in a proper way, there are a few tests that can be performed:
\begin{enumerate}
	\item Access to the Airflow UI and review the status of the last iterations of the ``\texttt{spark\_main}'' DAG.
	\item Access to the Spark UI (\url{http://localhost:8181}) and check that the worker nodes are executing or have executed the tasks sent from Airflow.
	\item Access to the \texttt{spark/resources/history} folder and check that the historic file has been created with the format ``\texttt{YYYYMMDDhhmmss}'' (year, month, day, hour, minute, second).
	\item Access to the ``\texttt{Visuals}'' view in the web application and visualize the metrics sorting the data by different columns. Then, access to the ``\texttt{Data}'' view and check that the data displayed is the same. E.g., sort by \texttt{energy} in descendant order in both views and check that the lists match.
\end{enumerate}
\apendice{User documentation}

\section{Introduction}

\section{User requirements}

\section{Installation}

\section{User's manual}

\bibliographystyle{unsrt}
\bibliography{bibliography}

\end{document}
