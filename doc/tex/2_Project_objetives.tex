\capitulo{2}{Project objectives}

\nonzeroparskip The initial objectives through which the use case was built consisted in the following points:
\begin{itemize}
	\item Ability to obtain data in real time.
	\item Combination of at least two different data sources.
	\item Potential to scale in both technology and data volume.
	\item Involvement of various technologies in the Big Data field.
	\item Use of open source tools.
\end{itemize}

\nonzeroparskip After a research, the author designed the use case and built the project objectives:
\begin{itemize}
	\item Build a pipeline to gather information about last songs listened from the \textbf{Twitter API}.
	\item Find information about the songs (name, artist and audio features) through the \textbf{Spotify API}.
	\item Execute all the ETL\footnote{ETL is the acronym for Extract, Transform and Load, the three phases for data processing} process in \textbf{Apache Spark}.
	\item Store all the data in a Data Warehouse under a known technology, \textbf{Apache Cassandra}.
	\item Visualize the information in a custom front-end and back-end created with \textbf{Flask} and \textbf{Bootstrap}.
	\item Orchestrate all the data flow with \textbf{Apache Airflow}.
	\item Develop the project with \textbf{Docker} and \textbf{Docker Compose} to ensure deployment through heterogeneous environments.
\end{itemize}

\nonzeroparskip Through these global objectives, the low-level functional and technical requirements were specified, as shown in the appendix \ref{requirements}. The detailed use case is described in the section \ref{analysis}.