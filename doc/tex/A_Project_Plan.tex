\apendice{Project Plan}

\section{Introduction}
\nonzeroparskip The project planning was decided in an initial meeting between the author and its tutor. It was based in an Agile methodology, with two-weeks \textit{sprints} and meetings between the author and his tutor conditioned to their availability.\\

\nonzeroparskip The project repository was stored in GitHub under the url \url{https://github.com/AdrianRiesco/Data-Engineer-project}. Each \textit{sprint} was created as an \textit{milestone}, with the \textit{issues} contained there being the tasks asigned. The \textit{issues} were created to reflect tasks at most eight hours, allowing the author segregate his work and manage each \textit{sprint} better. The author closed an \textit{issue} when the task was finished and a \textit{milestone} when the \textit{sprint} was over, regardless of its state. If a task remained in an open state when a \textit{sprint} reached its planned end date, the \textit{issue} was transfered to the next \textit{milestone}.\\

\nonzeroparskip A meeting was held by the author and his tutor at the end of each sprint. During these meetings, both of them reviewed the state and development of the tasks of the corresponding sprint and planned the tasks of the next sprint. All the \textit{milestones} and \textit{issues} can be consulted in the project repository.\\

\section{Temporary planning}
\nonzeroparskip The sprints carried out for the development of the project are described below with they correspondant dates:
\begin{description}
	\item[Initial meeting.] Held on Monday January 31st, it was the start point for the first sprint. During this meeting, the objective of the project, the data source and the tools to be used were validated by both the author and his tutor. The author previously made a research and came with an idea and the tutor exposed his point of view to create the final goal.
	\item[Sprint 1.] Weeks of January 31st and February 7th. This Sprint had the following tasks assigned:
	\begin{itemize}
		\item Configure the work environment.
		\item Configure the project memory template.
		\item Write a draft of the objectives and main goals.
		\item Write a brief description of the tools selected.
		\item Write a brief explanation of the selected tools and the work methodology.
		\item Inspect Twitter API.
		\item Inspect Spotify API.
	\end{itemize}
	The end-of-sprint meeting was held on Wednesday February 16th.
	Analysis: Most of the activities were realized by the author, excepting the Inspection of the Spotify API. Regarding the Twitter API, the author inspected the output and he concluded that it had the characteristics needed to be used to launch queries to the Spotify API (the tweet could be cleaned to get the song name and artist).
	\item[Sprint 2.] Weeks of February 14th and February 21st. This Sprint had the following tasks assigned:
	\begin{itemize}
		\item Task1.
	\end{itemize}
	The end-of-sprint meeting was held on M--- February --th.
	\item[Sprint 3]. Weeks of February 28th and March 7th. This Sprint had the following tasks assigned:
	\begin{itemize}
		\item Task1.
	\end{itemize}
	The end-of-sprint meeting was held on M--- March --th.
	\item[Sprint 4]. Weeks of March 14th and March 21st. This Sprint had the following tasks assigned:
	\begin{itemize}
		\item Task1.
	\end{itemize}
	The end-of-sprint meeting was held on M--- March --th.
	\item[Sprint 5]. Weeks of March 28th and April 4th. This Sprint had the following tasks assigned:
	\begin{itemize}
		\item Task1.
	\end{itemize}
	The end-of-sprint meeting was held on M--- April --th.
	\item[Sprint 6]. Weeks of April 11th and April 18th. This Sprint had the following tasks assigned:
	\begin{itemize}
		\item Task1.
	\end{itemize}
	The end-of-sprint meeting was held on M--- April --th.
	\item[Sprint 7]. Weeks of April 25th and May 2nd. This Sprint had the following tasks assigned:
	\begin{itemize}
		\item Task1.
	\end{itemize}
	The end-of-sprint meeting was held on M--- May --th.
	\item[Sprint 8]. Weeks of May 9th and May 16th. This Sprint had the following tasks assigned:
	\begin{itemize}
		\item Task1.
	\end{itemize}
	The end-of-sprint meeting was held on M--- May --th.
	\item[Sprint 9]. Weeks of May 23rd and May 30th. This Sprint had the following tasks assigned:
	\begin{itemize}
		\item Task1.
	\end{itemize}
	The end-of-sprint meeting was held on M--- June --th.
	\item[Sprint 10]. Weeks of June 6th and May 13th. This Sprint had the following tasks assigned:
	\begin{itemize}
		\item Task1.
	\end{itemize}
	The end-of-sprint meeting was held on M--- June --th.
\end{description}

\section{Feasibility study}
\nonzeroparskip The architecture of the project and the use case were designed to ensure its feasibility.

\subsection{Economic feasibility}
\nonzeroparskip The project is based on open-source platforms to ensure its economic and legal feasibility. The APIs where the information was gathered are free to use if the developer keeps his queries under specific limit rates.

\subsection{Legal feasibility}
\nonzeroparskip The project is based on open-source platforms to ensure its economic and legal feasibility. 

