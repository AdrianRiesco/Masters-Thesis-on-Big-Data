\apendice{Project Plan}

\section{Introduction}
\nonzeroparskip The project planning was decided in an initial meeting between the author and his tutor. It was based in an Agile methodology, with two-weeks \textit{sprints} and meetings between the author and his tutor conditioned to their availability.

\nonzeroparskip The project repository was stored in GitHub under the url \url{https://github.com/AdrianRiesco/Masters-Thesis-on-Big-Data}. Each \textit{sprint} was created as a \textit{milestone}, with the \textit{issues} contained there being the tasks assigned. The \textit{issues} were created to reflect tasks at most eight hours, allowing the author segregate his work and manage each \textit{sprint} better. The author closed an \textit{issue} when the task was finished and a \textit{milestone} when the \textit{sprint} was over, regardless of its state. If a task remained in an open state when a \textit{sprint} reached its planned end date, the \textit{issue} was transfered to the next \textit{milestone}.

\nonzeroparskip A meeting was held by the author and his tutor at the end of each sprint. During these meetings, both of them reviewed the state and development of the tasks of the corresponding sprint and planned the tasks of the next sprint. All the \textit{milestones} and \textit{issues} can be consulted in the project repository.

\section{Temporary planning}
\nonzeroparskip The sprints carried out for the development of the project are described below with their corresponding dates:
\begin{description}
	\item[Initial meeting.] Held on Monday, January 31\textsuperscript{st}, it was the start point for the first sprint. During this meeting, the objective of the project, the data source and the tools to be used were validated by both the author and his tutor. The author previously made a research and came with an idea and the tutor exposed his point of view to create the final goal.
	\item[Sprint 1.] Weeks of January 31\textsuperscript{st} and February 7\textsuperscript{th}. This Sprint had the following tasks assigned:
	\begin{itemize}
		\item Configure the work environment.
		\item Configure the project memory template.
		\item Write a draft of the objectives and main goals.
		\item Write a brief description of the tools selected.
		\item Write a brief explanation of the selected tools and the work methodology.
		\item Inspect Twitter API.
		\item Inspect Spotify API.
	\end{itemize}
	The end-of-sprint meeting was held on Wednesday, February 16\textsuperscript{th}.
	Analysis: Most of the activities were realized by the author, excepting the Inspection of the Spotify API. Regarding the Twitter API, the author inspected the output and he concluded that it had the characteristics needed to be used to launch queries to the Spotify API (the tweet could be cleaned to get the song name and artist).
	\item[Sprint 2.] Weeks of February 14\textsuperscript{th} and February 21\textsuperscript{st}. This Sprint had the following tasks assigned:
	\begin{itemize}
		\item Write the code to gather information from Spotify.
		\item Write the code to gather information from Twitter.
		\item Write a description of Spotify and Twitter APIs.
		\item Write the API inspection process in the ``Programmer guide'' section.
		\item Write the project introduction.
		\item Write the Twitter and Spotify data description.
	\end{itemize}
	The end-of-sprint meeting was held on Wednesday February 2nd. Analysis: All the activities were accomplished on time. The author identified some potential problems when extracting information from the Spotify API, such as the name of the song must be quite accurate to be able to get the search results.
	\item[Sprint 3.] Weeks of February, 28\textsuperscript{th} and March 7\textsuperscript{th}. This Sprint had the following tasks assigned:
	\begin{itemize}		
		\item Write a description of Twitter API.
		\item Write a description of Spotify API.
		\item Write the description of the Twitter data.
		\item Write the description of the Spotify data.
		\item Improve and unify the code written to collect data from both APIs.
		\item Set up the Spark environment using Docker.
	\end{itemize}
	The end-of-sprint meeting was held on Wednesday, March 16\textsuperscript{th}. Analysis: During this sprint, the author had difficulties configuring the Docker environment, which derived in a delay of the other tasks. These tasks were transfered to the next sprint.
	\item[Sprint 4.] Weeks of March 14\textsuperscript{th} and March 21\textsuperscript{st}. This Sprint had the following tasks assigned:
	\begin{itemize}
		\item Write a description of NoSQL Databases.
		\item Set up the Spark environment using Docker.
		\item Set up the Airflow environment using Docker.
	\end{itemize}
	The end-of-sprint meeting was held on Wednesday, May 4\textsuperscript{th}. Analysis: Due to personal reasons, the author could not continue with the project during the month of April, so there was a temporary pause in the planning.
	\item[Sprint 5.] Weeks of May 2\textsuperscript{nd} and May 9\textsuperscript{th}. This Sprint had the following tasks assigned:
	\begin{itemize}
		\item Configure the DAG in Airflow.
		\item Learn the fundamentals of Flask.
		\item Redesign the project plan excluding the month of April
	\end{itemize}
	The end-of-sprint meeting was held on Wednesday, May 18\textsuperscript{th}. Analysis: All tasks were accomplished on time.
	\item[Sprint 6.] Weeks of May 16\textsuperscript{th} and May 23\textsuperscript{rd}. This Sprint had the following tasks assigned:
	\begin{itemize}
		\item Deploy Cassandra environment.
		\item Write the description of the Twitter data.
		\item Write the description of the Spotify data.
		\item Write a description of an orchestrator.
		\item Write a description of CI/CD.
	\end{itemize}
	The end-of-sprint meeting was held on Monday, May 30\textsuperscript{th}. Analysis: All tasks were accomplished on time. The Cassandra deployment took most of Sprint's time, generating a few prototypes before the final version.
	\item[Sprint 7.] Weeks of May 30\textsuperscript{th} and June 6\textsuperscript{nd}. This Sprint had the following tasks assigned:
	\begin{itemize}
		\item Integrate Cassandra with Airflow and Spark.
		\item Create the ETL workflow to load the data to Cassandra.
		\item Write a description of Apache Spark.
		\item Write a description of Docker and Docker Compose.
		\item Write a description of Flask, Jinja and Bootstrap.
	\end{itemize}
	The end-of-sprint meeting was held on Tuesday, June 14\textsuperscript{th}. Analysis: All tasks were accomplished on time. The integration of Cassandra with Airflow and Spark took most of Sprint's time.
	\item[Sprint 8.] Weeks of June 13\textsuperscript{th} and June 20\textsuperscript{th}. This Sprint had the following tasks assigned:
	\begin{itemize}
		\item Create the front-end with Flask and Bootstrap.
		\item Integrate the ETL workflow with the front-end.
		\item Write section 5 ``Relevant aspects of the project''.
		\item Write section 7 ``Conclusions of the project''.
	\end{itemize}
	The end-of-sprint meeting was held on Monday, June 27\textsuperscript{th}. An additional meeting was held in the middle of the sprint to review and propose changes to the first version of the front-end. Analysis: Although the front-end development was very time-consuming, all tasks were accomplished on time. At this point, all development work was finished.
	\item[Sprint 9.] Weeks of June 27\textsuperscript{rd} and July 4\textsuperscript{th}. This Sprint had the following tasks assigned:
	\begin{itemize}
		\item Write section 6 ``Related works''.
		\item Complete Appendix A ``Project plan''.
		\item Complete Appendix B ``Requirements''.
		\item Complete Appendix C ``Design''.
		\item Complete Appendix D ``Programmer's guide''.
		\item Complete Appendix E ``User manual''.
		\item Review the entire project report.
	\end{itemize}
	The end-of-sprint meeting was held on Tuesday, July 12\textsuperscript{th}. An additional mid-sprint meeting was held to review the report and plan the final tasks to finalize the project for defense. Analysis: The report was finished and the project was properly delivered. During this week an Open Virtualization Appliance (OVA) and a presentation video were elaborated to be delivered to the review board so that they could execute the project locally. In addition, the project was uploaded to a Google Cloud virtual machine (Google Compute Engine service) to show an execution of the project without having to run it locally.
\end{description}

\section{Feasibility study}
\nonzeroparskip The architecture of the project and the use case were designed to ensure its feasibility.

\subsection{Economic feasibility}
\nonzeroparskip The project is based on open-source platforms to ensure its economic and legal feasibility. The APIs where the information was gathered are free to use if the developer keeps his queries under specific limit rates.

\subsection{Legal feasibility}
\nonzeroparskip The project is based on open-source platforms to ensure its economic and legal feasibility. The licenses in which the technology and tools used in the project are based are:
\begin{itemize}
	\item \textbf{Twitter License.} Free with usage limitations and requires developer account~\cite{twitter_dev_license}.
	\item \textbf{Spotify License.} Free with usage limitations and requires developer account~\cite{spotify_dev_license}.
	\item \textbf{Apache License.} Free with limitations~\cite{apache_license}.
	\item \textbf{Flask License.} Free with limitations, uses BSD-3-Clause license~\cite{flask_license}.
	\item \textbf{Bootstrap License.} Free with limitations, uses Massachusetts Institute of Technology (MIT) license~\cite{bootstrap_license}.
	\item \textbf{\TeX{}maker License.} Free with limitations, uses GNU General Public License (GPL) license~\cite{texmaker}.
\end{itemize}

\tablaSmall{Tools and technologies' licenses}{l c c c c c c}{licensespertool}
{ \multicolumn{1}{l}{Tools} & Twitter & Spotify & Apache & BSD-3-Clause & MIT & GPL\\}{ 
Twitter API & X & & & & &\\
Spotify API & & X & & & &\\
Apache Airflow & & & X & & &\\
Apache Spark & & & X & & &\\
Apache Cassandra & & & X & & &\\
Flask & & & & X & &\\
Bootstrap & & & & & X &\\
Datatables & & & & & X &\\
Chart.js & & & & & X &\\
\TeX{}Maker & & & & & & X\\
}