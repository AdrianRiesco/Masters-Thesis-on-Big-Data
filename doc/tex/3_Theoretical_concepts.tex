\capitulo{3}{Theoretical concepts}

%Theoretical concepts of \LaTeX \footnote{Example of footnote}.
%\subsection{Subsection}
%\subsubsection{Subsubsection}
%Use of cite \cite{wiki:latex}, \cite{koza92}.
%\imagen{img/escudoInfor}{Image caption}
%Lists
%\begin{itemize}
%Enumerate.
%\begin{enumerate}
%Description.
%\begin{description}

\nonzeroparskip In this section are covered the theoretical concepts in which the project has been based. All concepts are described in a detailed an simple way since this master's degree can be aimed at technical and non-technical students.

\section{API}

\nonzeroparskip An Aplication Interface or API is Section explaining API concepts.

\subsection{Twitter API}
\nonzeroparskip Section explaining Twitter.
\nonzeroparskip Section explaining Twitter API.

\subsection{Spotify API}
\nonzeroparskip Section explaining Spotify.
\nonzeroparskip Section explaining Spotify API.

\section{Orchestrator}

\nonzeroparskip Section explaining Flow Orchestrator -> Airflow.

\section{NoSQL Databases}

\nonzeroparskip Section explaining NoSQL Databases.

\section{Containers}

\nonzeroparskip Section explaining Containers.

\section{Continuous Integration / Continuous Delivery}

\nonzeroparskip Section explaining CI/CD.

\section{Template engines}

\nonzeroparskip Section explaining Template engines -> Jinja.

\section{Web Server Gateway Interface}

\nonzeroparskip Section explaining Web Server Gateway Interface (WSGI).


\section{Tables}

\nonzeroparskip TablaSmall.

\tablaSmall{Tools and technologies used}{l c c c c}{herramientasportipodeuso}
{ \multicolumn{1}{l}{Tools} & App AngularJS & API REST & BD & Memoria \\}{ 
HTML5 & X & & &\\
CSS3 & X & & &\\
BOOTSTRAP & X & & &\\
\TeX{}Maker & & & & X\\
Astah & & & & X\\
} 
