\apendice{Requirements} \label{requirements}

\section{Introduction}

\nonzeroparskip This section lists the general objectives and requirements identified during the initial planning of the project and on whose fulfillment the development of the project has focused.

\section{General objectives}

\nonzeroparskip The requirements through which the use case was built were the following:
\begin{itemize}
	\item Ability to obtain data in real time.
	\item Combination of at least two different data sources.
	\item Potential to scale in both technology and data volume.
	\item Involvement of various technologies in the Big Data field.
	\item Use of open source tools.
\end{itemize}

\section{Catalog of requirements}

\subsection{Functional requirements}

\nonzeroparskip The functional requirements (FR) that the project had to meet were:
\begin{itemize}
	\item \textbf{FR1}. The data must be obtained from the Twitter hashtag \texttt{\textit{\#NowPlaying}} every 30 minutes, taking care of API rate limits.
	\begin{itemize}
		\item \textbf{Status}: Met.
		\item \textbf{Rationale}: The Apache Airflow DAG was configured with a run frequency of 30 minutes. The script used to collect the data has a maximum of 100 tweets collected per run to comply with API rate limits.
	\end{itemize}
	\item \textbf{FR2}. There must be at least two different visualizations and one of them must provide the ability to view all of the stored data.
	\begin{itemize}
		\item \textbf{Status}: Met.
		\item \textbf{Rationale}: There are two visualizations in the web application, ``Data'' and ``Visuals'', and the first one displays all the stored data.
	\end{itemize}
	\item \textbf{FR3}. At least one of the visualizations must show last songs name, artist and audio features.
	\begin{itemize}
		\item \textbf{Status}: Met.
		\item \textbf{Rationale}: The ``Data'' view of the web application displays all required fields.
	\end{itemize}
	\item \textbf{FR4}. At least one of the visualizations must have a link to the source tweet.
	\begin{itemize}
		\item \textbf{Status}: Met.
		\item \textbf{Rationale}: The ``Data'' view of the web application contains a link to the source tweet.
	\end{itemize}
	\item \textbf{FR5}. At least one of the visualizations must have the ability to compare different metrics.
	\begin{itemize}
		\item \textbf{Status}: Met.
		\item \textbf{Rationale}: The ``Visuals'' view of the web application provides the ability to visually compare metrics.
	\end{itemize}
	\item \textbf{FR6}. At least one of the visualizations must combine two different types of visualizations.
	\begin{itemize}
		\item \textbf{Status}: Met.
		\item \textbf{Rationale}: The ``Visuals'' view of the web application combines bar and line formats in the same chart.
	\end{itemize}
	\item \textbf{FR7}. Both visualizations must provide sorting capabilities.
	\begin{itemize}
		\item \textbf{Status}: Met.
		\item \textbf{Rationale}: Both ``Data'' and ``Visuals'' views of the web application provides provides the ability to sort by metric.
	\end{itemize}
	\item \textbf{FR8}. Both visualizations must be responsive to different screen sizes.
	\begin{itemize}
		\item \textbf{Status}: Met.
		\item \textbf{Rationale}: The web application was developed using Bootstrap as the CSS framework to facilitate scaling and resizing on different screens.
	\end{itemize}
\end{itemize}

\subsection{Technical requirements}

\nonzeroparskip The technical requirements (TR) that the project had to meet were:
\begin{itemize}
	\item \textbf{TR1}. The development must have the ability to be deployed in different environments with minimum effort.
	\begin{itemize}
		\item \textbf{Status}: Met.
		\item \textbf{Rationale}: The services were developed using multiple containers managed via Docker Compose.
	\end{itemize}
	\item \textbf{TR2}. The data flow must be automated, with the entire process orchestrated by a single tool.
	\begin{itemize}
		\item \textbf{Status}: Met.
		\item \textbf{Rationale}: All data flow is orchestrated by Apache Airflow.
	\end{itemize}
	\item \textbf{TR3}. The execution of the ETL process must be done with a tool that can scale and run in distributed environments.
	\begin{itemize}
		\item \textbf{Status}: Met.
		\item \textbf{Rationale}: Data extraction, transformation and loading (into a .csv file) is processed by Apache Spark, and data loading to Cassandra is performed by a Cassandra Query Language shell (cqlsh) command.
	\end{itemize}
	\item \textbf{TR4}. The data warehouse must have the ability to escalate in terms of a Big Data problem.
	\begin{itemize}
		\item \textbf{Status}: Met.
		\item \textbf{Rationale}: Apache Cassandra is the Data Warehouse used.
	\end{itemize}
	\item \textbf{TR5}. The web application must be designed with widely recognized tools.
	\begin{itemize}
		\item \textbf{Status}: Met.
		\item \textbf{Rationale}: The web application was designed with Flask and Bootstrap, and the visualizations with Chart.js and Datatables.
	\end{itemize}
	\item \textbf{TR6}. All the tools used must be open source.
	\begin{itemize}
		\item \textbf{Status}: Met.
		\item \textbf{Rationale}: All the tools used to develop the project are open source.
	\end{itemize}
\end{itemize}