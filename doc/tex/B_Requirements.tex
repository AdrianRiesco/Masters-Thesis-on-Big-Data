\apendice{Requirements}

\section{Introduction}

\section{General objectives}

\nonzeroparskip The requirements through which the use case was built were the following:
\begin{itemize}
	\item Ability to obtain data in real time.
	\item Combine at least two different data sources.
	\item Potential to scale in both technology and data volume.
	\item Involve various technologies from the world of Big Data.
	\item Mostly open source tools.
\end{itemize}

\section{Catalog of requirements}

\nonzeroparskip The functional requirements that the project had to meet were:
\begin{description}
	\item[F1] The data must be obtained from the Twitter hashtag \#NowPlaying every 15 minutes, taking care of API rate limits.
	\item[F2] The visualizations must show last songs name, artist and audio features.
	\item[F3] The visualization must have a link to the source tweet.
	\item[F4] -
\end{description}

\nonzeroparskip The technical requirements that the project had to meet were:
\begin{description}
	\item[T1] Ability to be deployed in different environments with minimum effort.
	\item[T2] Automated data flow, with hole process orchestrated by a unique tool.
	\item[T3] Data warehouse with ability to escalate in terms of a Big Data problem.
	\item[T4] -
\end{description}

\section{Requirements specification}


