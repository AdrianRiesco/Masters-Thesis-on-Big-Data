\capitulo{5}{Relevant aspects of the project}

%This section aims to collect the most interesting aspects of the development of the project, commented by its authors.
%It must include from the exposition of the life cycle used, to the most relevant details of the analysis, design and implementation phases.
%It is sought that it is not a mere operation of copying and pasting diagrams and extracts from the source code, but that the solution paths that have been taken are really justified, especially those that are not trivial.
%It may be the most appropriate place to document the most interesting aspects of the design and implementation, with a greater emphasis on aspects such as the type of architecture chosen, the indexes of the database tables, normalization and denormalization, distribution in files3, business rules within databases (EDVHV GH GDWRV DFWLYDV), development aspects related to the WWW...
%This section must become the summary of the practical experience of the project, and by itself justifies that the report becomes a useful document, a reference source for authors, tutors and future students.

\nonzeroparskip The first step of the project was the feasibility and viability analysis of the concept devised. The author was looking to use two data sources with:

\begin{itemize}
	\item Real and updated data, preferable related to the social interest.
	\item The possibility of getting a stream data flow.
	\item The potential to combine both to get an added value.
\end{itemize}

\nonzeroparskip Considering the previous points, the author found an interesting option on Twitter and Spotify providers. Both of them provides solid APIs for a fluid development and have the characteristics needed to combine the data collected. Consequently, the author designed the following use case:

\begin{enumerate}
	\item The Twitter API is consulted to gather the \textit{tweets} with the hashtag \textit{\#NowPlaying}.
	\item The tweet is cleaned, removing the stopwords and the other hashtags and getting the song name and artist as isolated as possible.
	\item The Spotify API is consulted to gather the information of the song identified.
	\item The vector values of the cleaned Twitter data and the name of the song returned by Spotify are compared to ensure they are the same.
	\item The data is moved to the database, ready to be stored and visualized.
\end{enumerate}

\nonzeroparskip During the design phase, the author analyzed the output of both APIs using Postman. In the first place, relying on the Twitter documentation, the author inspected the Twitter API by following the next steps:
\begin{enumerate}
	\item Get access to the Twitter Developer Portal.
	\item Get the credentials needed to consult the different endpoints of the API.
	\item Import the \textit{Twitter API v2} collection on Postman.
	\item Modify the automatically created environment called \textit{Twitter API v2} to include the following developer keys and tokens:
	\begin{itemize}
		\item Consumer key (consumer\_key).
		\item Consumer secret (consumer\_secret).
		\item Access token (access\_token).
		\item Token secret (token\_secret).
		\item Bearer token (bearer\_token).
	\end{itemize}
	\item In the collection tab, select the endpoint \textit{Tweet Lookup} -> \textit{Single Tweet} for the initial exploration.
	\item Create a fork and configure the following parameters:
	\begin{itemize}
		\item[tweet.fields] 
		\item[expansions]
		\item[id] - The identifier of a tweet similar to the ones to gather. It can be find in the tweet's URL.
		\item Token secret (token\_secret).
		\item Bearer token (bearer\_token).
	\end{itemize}
	\item Get access to the Twitter Developer Portal.
\end{enumerate}


\nonzeroparskip The project development was undertaken following an Agile methodology.\\