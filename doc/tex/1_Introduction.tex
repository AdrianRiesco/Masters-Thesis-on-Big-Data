\capitulo{1}{Introduction}

% Description of the work, the structure of the memory and the rest of the material delivered.

\nonzeroparskip Social networks are currently a fundamental aspect of society. People usually use social networks to share experiences, opinions and aspects of their lives and interact with other people. Using social networks as a data source we can access a huge amount of information and be able to build accurate analysis on practically any topic.

\nonzeroparskip On the other hand, another aspect that has gradually permeated our society is the concept of subscriptions to services, be it music, movies, games or almost any concept that we can think of. Not so many years ago, the concept of paying for subscriptions to services, where you do not actually own the content you pay for and instead get temporary access whose duration is defined by how long you continue to pay for the subscription, was relegated to very specific services and was not nearly as globalized as it is today.

\nonzeroparskip The global acceptance of subscription as a service is reflected in social networks, where users can comment on the different music, movie and game platforms, each new development becoming a social phenomenon. With the aim of taking advantage of both worlds, in this project the social network Twitter will be used to obtain information on the latest music listened to by users (by searching for a certain hashtag) and subsequently consult the data of the song and the artist involved that Spotify, a platform based on music as a service, has. The development of the project has followed an agile methodology with two-week sprints.

%\nonzeroparskip Other technologies used.