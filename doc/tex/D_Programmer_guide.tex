\apendice{Programming technical documentation}

\section{Introduction}

\section{Directory structure}
\nonzeroparskip The project repository, hosted on GitHub, has the following directory structure:
\begin{itemize}
	\item \textbf{\texttt{airflow}}.
	\item \textbf{\texttt{doc}}. It contains the project report.
	\item \textbf{\texttt{docker}}. 
	\item \textbf{\texttt{spark}}. 
	\item \textbf{\texttt{README.md}}. 
\end{itemize}

\section{Programmer's guide}

\subsection{Analysis}

\nonzeroparskip During the analysis phase, the author inspected the output of Twitter and Spotify APIs using Postman. In the first place, relying on the Twitter documentation, the author inspected the Twitter API by following the next steps:
\begin{enumerate}
	\item Get access to the Twitter Developer Portal.
	\item Get the credentials needed to consult the different endpoints of the API.
	\item Import the \textit{Twitter API v2} collection on Postman.
	\item Create a fork of the automatically created environment (\textit{Twitter API v2}) and collection \textit{Twitter API v2} to be able to edit the values.
	\item Modify the environment to include the following developer keys and tokens:
	\begin{itemize}
		\item Consumer key (\texttt{consumer\_key}).
		\item Consumer secret (\texttt{consumer\_secret}).
		\item Access token (\texttt{access\_token}).
		\item Token secret (\texttt{token\_secret}).
		\item Bearer token (\texttt{bearer\_token}).
	\end{itemize}
	\item In the collection tab, select the endpoint \textit{Search Tweets} $\longrightarrow$ \textit{Recent search} for the initial exploration. Configure the following parameters:
	\begin{itemize}
		\item \texttt{query} = \texttt{\#NowPlaying}
		\item \texttt{tweet.fields} = \texttt{created\_at,entities}
		\item \texttt{max\_results} = \texttt{10}
	\end{itemize}
	\item Now, we can send our query \url{https://api.twitter.com/2/tweets/search/re cent?query=\%23NowPlaying\&max\_results=10\&tweet.fields=created\_a t,entities} to get the 10 most recent tweets with the hashtag \texttt{\textit{\#NowPlaying}} and receive their basic information (id, text) as well as the entities (hashtags, urls, annotations...) and the creation time stamp.
\end{enumerate}

\nonzeroparskip After analyze the data gathered from the Twitter API,  the author inspected the Spotify API (more specifically the endpoint ``Search for Item''), following the next steps:
\begin{enumerate}
	\item Get access to the Spotify Developer Portal.
	\item Enter in the developers console and select the ``Search for Item'' endpoint.
	\item Specify the parameters of the search. We can specify the type ``track'' and add a limit of one to only receive the first song found.
	\item After click on get the bearer token, we can use that token by clicking on Try Me or just copy the resulting query in our Linux console to check the output.
	\item The result of this query is the first result of the search containing information of the artist, the song and the album. With the artist and song ids we can consult other endpoints to get an audio analysis, the audio features and the artist information, between others.
\end{enumerate}

\subsection{Development}
\nonzeroparskip During the development phase, the following items were installed in the system:
\begin{itemize}
	\item Docker version 20.10.12, build e91ed57.
	\item docker-compose version 1.29.2, build 5becea4c.
	\item docker-compose version 1.29.2, build 5becea4c.
\end{itemize}

\nonzeroparskip Steps followed:
\begin{enumerate}
	%\item Create the dockerfile.
	%\item Build the image:  \texttt{docker build -f spark.Dockerfile -t ar/spark-ubuntu .}
	%\item Create a Docker network: \texttt{docker network create --driver bridge spark-network}
	%\item Run the image: \texttt{sudo docker run -d -t --name master --network spark-network ar/spark-ubuntu}
	\item Create the Docker Compose file.
	\item Launch the environment with \texttt{sudo docker-compose up --remove-orphans}.
	\item ***Build the extended image: \texttt{sudo docker build . -f Dockerfile --pull --tag extended/airflow:2.3.0}.
	\item ***Build the extended image: \texttt{sudo docker-compose build}.
	\item ***Execute airflow-init: \texttt{sudo docker-compose up airflow-init}.
	\item ***\texttt{sudo docker-compose up}.
	\item ***In Airflow UI, create Spark connection.
\end{enumerate}

\section{Compilation, installation and execution of the project}
\nonzeroparskip The instructions to execute the project are the following ones:
\begin{enumerate}
	\item Create the Docker Compose file.
	\item Launch the environment with \texttt{sudo docker-compose up --remove-orphans}.
\end{enumerate}

\section{System tests}
\nonzeroparskip To test that the project is working in a proper way, there are a few tests that can be performed:
\begin{enumerate}
	\item Test 1.
	\item Test 2.
	\item Test 3.
\end{enumerate}