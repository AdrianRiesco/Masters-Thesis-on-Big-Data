\apendice{Programming technical documentation}

\section{Introduction}

\section{Directory structure}
Estructura GitHub.

\section{Programmer's guide}

\subsection{Analysis}

\nonzeroparskip During the analysis phase, the author inspected the output of Twitter and Spotify APIs using Postman. In the first place, relying on the Twitter documentation, the author inspected the Twitter API by following the next steps:
\begin{enumerate}
	\item Get access to the Twitter Developer Portal.
	\item Get the credentials needed to consult the different endpoints of the API.
	\item Import the \textit{Twitter API v2} collection on Postman.
	\item Modify the automatically created environment called \textit{Twitter API v2} to include the following developer keys and tokens:
	\begin{itemize}
		\item Consumer key (consumer\_key).
		\item Consumer secret (consumer\_secret).
		\item Access token (access\_token).
		\item Token secret (token\_secret).
		\item Bearer token (bearer\_token).
	\end{itemize}
	\item In the collection tab, select the endpoint \textit{Tweet Lookup} -> \textit{Single Tweet} for the initial exploration.
	\item Create a fork and configure the following parameters:
	\begin{itemize}
		\item[tweet.fields] 
		\item[expansions]
		\item[id] - The identifier of a tweet similar to the ones to gather. It can be find in the tweet's URL.
		\item Token secret (token\_secret).
		\item Bearer token (bearer\_token).
	\end{itemize}
	\item Get access to the Twitter Developer Portal.
\end{enumerate}

\subsection{Development}

\section{Compilation, installation and execution of the project}


\section{System tests}

