\apendice{Programming technical documentation}

\section{Introduction}

\nonzeroparskip This section contains the directory structure used and the main technical details that a user wishing to reproduce or execute the project should be aware of.

\section{Directory structure}
\nonzeroparskip The project repository, hosted on GitHub, has the following directory structure:
\begin{itemize}
	\item \textbf{\texttt{airflow}}. It contains the created DAG and the folders that Airflow needs to function properly.
	\item \textbf{\texttt{cassandra}}. I contains the database schema required.
	\item \textbf{\texttt{doc}}. It contains the project report and the \LaTeX{} files used to generate it.
	\item \textbf{\texttt{docker}}. It contains the Docker Compose file required to deploy the project and the Airflow, Spark and Flask folders used for building custom images.
	\item \textbf{\texttt{spark}}. It contains the script used to collect the API data, the necessary libraries and the history files.
\end{itemize}

\nonzeroparskip The directory structure is shown in the figure~\ref{directory}.

\figuraNormalSinMarco{0.40}{img/directory}{Directory structure}{directory}{}

\section{Programmer's guide}

\subsection{Analysis}

\nonzeroparskip During the analysis phase, the author inspected the output of Twitter and Spotify APIs using Postman. In the first place, relying on the Twitter documentation, the author examined the Twitter API by following the next steps:
\begin{enumerate}
	\item Get access to the Twitter Developer Portal.
	\item Get the credentials needed to consult the different endpoints of the API.
	\item Import the \textit{Twitter API v2} collection on Postman.
	\item Create a fork of the automatically created environment (\textit{Twitter API v2}) and collection \textit{Twitter API v2} to be able to edit the values.
	\item Modify the environment to include the following developer keys and tokens:
	\begin{itemize}
		\item Consumer key (\texttt{consumer\_key}).
		\item Consumer secret (\texttt{consumer\_secret}).
		\item Access token (\texttt{access\_token}).
		\item Token secret (\texttt{token\_secret}).
		\item Bearer token (\texttt{bearer\_token}).
	\end{itemize}
	\item In the collection tab, select the endpoint \textit{Search Tweets} $\longrightarrow$ \textit{Recent search} for the initial exploration. Configure the following parameters:
	\begin{itemize}
		\item \texttt{query} = \texttt{\#NowPlaying}
		\item \texttt{tweet.fields} = \texttt{created\_at,entities}
		\item \texttt{max\_results} = \texttt{10}
	\end{itemize}
	\item Now, the query can be sent \url{https://api.twitter.com/2/tweets/search/re cent?query=\%23NowPlaying\&max\_results=10\&tweet.fields=created\_a t,entities} to get the 10 most recent tweets with the hashtag \texttt{\textit{\#NowPlaying}} and receive their basic information (id, text) as well as the entities (hashtags, urls, annotations...) and the creation time stamps.
\end{enumerate}

\nonzeroparskip After analyze the data gathered from the Twitter API,  the author inspected the Spotify API (more specifically the endpoint ``Search for Item''), following the next steps:
\begin{enumerate}
	\item Get access to the Spotify Developer Portal.
	\item Enter in the developers console and select the ``Search for Item'' endpoint.
	\item Specify the parameters of the search. The type ``track'' can be specified and a limit of one can be added to receive only the first song found.
	\item After clicking get carrier token, that token can be used by clicking Try Me or by copying the resulting query into a Linux console to test the output.
	\item The result of this query is the first search result containing artist, song and album information. With the artist and song ids, other endpoints can be queried to obtain audio analysis, audio characteristics and artist information, among others.
\end{enumerate}

\subsection{Development} \label{programmer_development}
\nonzeroparskip To run the project, the following items must be installed in the system:
\begin{itemize}
	\item Docker\footnote{\url{https://docs.docker.com/engine/install/ubuntu/}}. The author used the version \texttt{20.10}.
	\item Docker Compose\footnote{\url{https://docs.docker.com/compose/install/}}. The author used the version \texttt{1.29}.
	\item Python\footnote{\url{https://docs.python-guide.org/starting/install3/linux/}}. The author used the version \texttt{3.8}.
\end{itemize}

\nonzeroparskip The above packaged were installed in a ``\texttt{Ubuntu 20.04.4 LTS}'' virtual machine. In case of using a different operating system, please refer to the package documentation.

\nonzeroparskip It is important to note that to run or play the project, the user has to modify the \texttt{.env} file located within the \texttt{docker} folder and include the respective developer keys for both Twitter and Spotify.

\fbox{\parbox{0.9\textwidth}{\textit{The commands described for the installation and usage of the project are oriented to Linux environments.}}}

\section{Compilation, installation and execution of the project} \label{programmer_execute}
\nonzeroparskip The steps needed to run the project are:
\begin{enumerate}
	\item Ensure that the folders \texttt{spark/resources/history}, \texttt{airflow/logs} and \texttt{airflow/plugins} have the necessary permissions by executing the command \texttt{sudo chmod -R 777 folder\_to\_set\_permissions}:
	\begin{itemize}
		\item \texttt{sudo chmod -R 777 spark/resources/history}
		\item \texttt{sudo chmod -R 777 airflow/logs}
		\item \texttt{sudo chmod -R 777 airflow/plugins}
	\end{itemize}
	\item Open a \texttt{command prompt} and move to the \texttt{docker} folder.
	\item Launch the environment with ``\texttt{sudo docker-compose up --build}'' (\texttt{--build} ensures that all the images are built during the deployment). It can be launched in the background by adding the flag \texttt{-d}: ``\texttt{sudo docker-compose up --build -d}''.
	\item Wait until all the services have started and are in a healthy state. It usually is achieved when the \texttt{airflow\_webserver\_container} is continuously displaying a status message. If the flag \texttt{-d} was included, the logs can be accessed with the command ``\texttt{sudo docker-compose logs}''. A list of the services can be displayed with the command ``\texttt{sudo docker ps}''.
	\item Access to the Airflow UI (\url{http://localhost:8080}, user ``\texttt{airflow}'' and password ``\texttt{airflow}'') and start the DAG ``\texttt{spark\_main}'' as shown in the figure~\ref{start-dag}.
	\item Access to the web application (\url{http://localhost:8000}) and check that the ``\texttt{Data}'' and ``\texttt{Visuals}'' views contain the captured data.
	\item To stop the containers, the command ``\texttt{sudo docker-compose down}'' can be used. To remove the networks and docker volumes, ``\texttt{sudo docker-compose down -v}'' must be used. The process can also be stopped by pausing the DAG as explained in step 5.
\end{enumerate}

\figuraNormalSinMarco{0.25}{img/start-dag}{Pause/unpause DAG}{start-dag}{}

\nonzeroparskip The author faced some problems during the development phase that are easily solved:
\begin{itemize}
	\item If the build fails for an error in the Airflow container related with the \texttt{/opt/airflow/logs} folder, it will probably be due to the permissions in the \texttt{airflow/logs} folder. To solve it, open a command prompt in the \texttt{airflow} folder and execute the commands ``\texttt{sudo chmod -R 777 logs}'' and ``\texttt{sudo chmod -R 777 plugins}''.
	\item If the DAG fails, the logs can be consulted in the \texttt{airflow/logs/dag\_id =spark\_main} folder. If the ``\texttt{cassandra\_load}'' task is failing, it can be due to the permissions of the \texttt{spark/resources/history} folder. To solve that, open a command prompt in the \texttt{spark/resources} folder and execute the command ``\texttt{sudo chmod -R 777 history}''. Then, remove the ``\texttt{data.csv}'' file located in the \texttt{spark/resources} folder (or manually create the history file) and execute the DAG again.
\end{itemize}

\section{System tests}
\nonzeroparskip To test that the project is working in a proper way, there are a few tests that can be performed:
\begin{enumerate}
	\item Access to the Airflow UI and review the status of the last iterations of the ``\texttt{spark\_main}'' DAG.
	\item Access to the Spark UI (\url{http://localhost:8181}) and check that the worker nodes are executing or have executed the tasks sent from Airflow.
	\item Access to the \texttt{spark/resources/history} folder and check that the historic file has been created with the format ``\texttt{YYYYMMDDhhmmss}'' (year, month, day, hour, minute, second).
	\item Access to the ``\texttt{Visuals}'' view in the web application and visualize the metrics sorting the data by different columns. Then, access to the ``\texttt{Data}'' view and check that the data displayed is the same. E.g., sort by \texttt{energy} in descendant order in both views and check that the lists match.
\end{enumerate}