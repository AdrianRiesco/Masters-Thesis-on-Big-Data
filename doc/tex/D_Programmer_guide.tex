\apendice{Programming technical documentation}

\section{Introduction}

\section{Directory structure}
Estructura GitHub.

\section{Programmer's guide}

\subsection{Analysis}

\nonzeroparskip During the analysis phase, the author inspected the output of Twitter and Spotify APIs using Postman. In the first place, relying on the Twitter documentation, the author inspected the Twitter API by following the next steps:
\begin{enumerate}
	\item Get access to the Twitter Developer Portal.
	\item Get the credentials needed to consult the different endpoints of the API.
	\item Import the \textit{Twitter API v2} collection on Postman.
	\item Create a fork of the automatically created environment (\textit{Twitter API v2}) and collection \textit{Twitter API v2} to be able to edit the values.
	\item Modify the environment to include the following developer keys and tokens:
	\begin{itemize}
		\item Consumer key (consumer\_key).
		\item Consumer secret (consumer\_secret).
		\item Access token (access\_token).
		\item Token secret (token\_secret).
		\item Bearer token (bearer\_token).
	\end{itemize}
	\item In the collection tab, select the endpoint \textit{Search Tweets} -> \textit{Recent search} for the initial exploration. Configure the following parameters:
	\begin{itemize}
		\item[query] \char23 NowPlaying
		\item[tweet.fields] created_at,entities
		\item[max_results] 10
	\end{itemize}
	\item Now, we can send our query \textit{https://api.twitter.com/2/tweets/search/recent?query=\%23NowPlaying&max_results=10&tweet.fields=created_at,entities} to get the 10 most recent tweets with the hashtag \char23 NowPlaying and receive the basic information (id, text) as well as entities (hashtags, urls, annotations...) and creation time stamp.
\end{enumerate}

\subsection{Development}

\section{Compilation, installation and execution of the project}


\section{System tests}

