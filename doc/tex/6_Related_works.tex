\capitulo{6}{Related works}

\nonzeroparskip Both the software tools used in the project, widely recognized within the Big Data industry, and the data sources, from large cap and renowned companies, have been applied to countless projects throughout their lifetime.

\nonzeroparskip During the planning and study phase of the APIs and tools, the author discovered several projects that have combined both elements in different ways. Although no project has been found that combines both APIs and may involve a data flow similar to that of the present project, the author has decided to mention those that have been considered most interesting either for their data flow or for the originality of the idea:

\begin{itemize}
	\item \textbf{Tweetpy}~\footnote{\url{https://www.tweepy.org/}}. Created in 2009 by Joshua Roesslein, this project contributes a Python library to interact with the Twitter API, simplifying interactions during development. Despite considering it as an alternative, the author decided not to make use of it in order to learn directly how to interact with the API. However, in the case of having to interact with user accesses, it would probably have been chosen.
	\item \textbf{Spotipy}~\footnote{\url{https://spotipy.readthedocs.io/}}. Created in 2014 by Paul Lamere, this project provides a Python library to interact with the Spotify API. As with the previous library, it was decided not to use it to learn from direct interaction with the API, since the data flow of the project does not involve complex access permissions.
	\item \textbf{Divide for Spotify}~\footnote{\url{https://divideforspotify.com/}}. This project makes use of the Spotify API to divide the songs that have been liked among the playlists of the user in question. This project has been chosen for its originality, as it takes advantage of the API to add a very interesting functionality to the user of the platform.
	\item \textbf{Spotify to Twitter}~\footnote{\url{https://github.com/transitive-bullshit/spotify-to-twitter}}. This project combines both APIs to automate a Twitter account to publish songs from a Spotify playlist.
	\item \textbf{}~\footnote{\url{}}.  **PENDIENTE COMPLETAR**
\end{itemize}

Trabajos similares con conjuntos de datos o flujo de datos similar. Hashtags, músicas... Herramientas o webs que hagan esos. 2 o 3 herramientas. Análisis de Twitter. Twitter trending topic analysis. Pueden ser artículos científicos.