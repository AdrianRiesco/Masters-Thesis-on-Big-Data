\capitulo{6}{Related works}

\nonzeroparskip Both the software tools used in the project, widely recognized within the Big Data industry, and the data sources, from large cap and renowned companies, have been applied to countless projects throughout their lifetime.

\nonzeroparskip During the planning and study phase of the APIs and tools, the author discovered several projects that have combined both elements in different ways. Although no project has been found that combines both APIs and involves a data flow similar to that of the present project, the author has decided to mention those considered most interesting, either because of their data flow, their usefulness or the originality of the idea:

\begin{itemize}
	\item \textbf{Tweetpy}~\footnote{\url{https://www.tweepy.org/}}. Created in 2009 by Joshua Roesslein, this project contributes a Python library to interact with the Twitter API, simplifying interactions during development. Despite considering it as an alternative, the author decided not to make use of it in order to learn directly how to interact with the API. However, in the case of having to interact with user accesses, it would probably have been chosen.
	\item \textbf{Spotipy}~\footnote{\url{https://spotipy.readthedocs.io/}}. Created in 2014 by Paul Lamere, this project provides a Python library to interact with the Spotify API. As with the previous library, it was decided not to use it to learn from direct interaction with the API, since the data flow of the project does not involve complex access permissions.
	\item \textbf{SpotaTweet}~\footnote{\url{https://github.com/twitterdev/spotatweet}}. Although its data flow is not analogous to that of the project, its use case bears some similarity, with a collection of tweets with the hashtag \texttt{\#NowPlaying} and a subsequent search for the song listened to on Spotify with the goal of displaying a preview to the user.
	\item \textbf{Divide for Spotify}~\footnote{\url{https://divideforspotify.com/}}. This project makes use of the Spotify API to divide the songs that have been liked among the playlists of the user in question. This project has been chosen for its originality, as it takes advantage of the API to add a very interesting functionality to the platform, automatically dividing the songs of a playlist to choose from a selection of playlists chosen by the user.
	\item \textbf{Spotify to Twitter}~\footnote{\url{https://github.com/transitive-bullshit/spotify-to-twitter}}. This project combines both APIs to automate a Twitter account to publish tweets with songs from a Spotify playlist. The project is based on \texttt{Node.js}, an open source, cross-platform execution environment for the JavaScript-based server layer.
\end{itemize}

\nonzeroparskip The projects mentioned above are only a fragment of those encountered during the initial analysis. Due to the breadth of Big Data domains, the use cases are innumerable and specialized tools can be of use in a number of different environments.